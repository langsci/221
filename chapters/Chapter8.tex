\chapter{Summary and Discussion} \label{Conclusion}

In this book, I have investigated \is{morphological gemination}{gemination} with the five English affixes \prefix{un}, \is{negative in-}negative \prefix{in}, \is{locative in-}locative \prefix{in}, \is{dis-}\prefix{dis} and \is{-ly}\suffix{ly}. I conducted a corpus and an experimental study. 
\tabref{tbl:Overview gemination in corpus and experimental study} summarizes the main results of both studies by displaying the \isi{gemination} pattern of each affix. Note that the \isi{gemination} pattern of locative and \is{negative in-}negative \prefix{in} is not displayed separately as there were no differences in \isi{gemination} between the two affixes. 





\begin{table*}
	\caption{Overview of gemination in corpus and experimental study}
	\label{tbl:Overview gemination in corpus and experimental study}
	
 
		\renewcommand{\arraystretch}{1.5}
		%\newcolumntype{L}[1]{>{\raggedright\arraybackslash}p{#1}}
		%\newcolumntype{R}[1]{>{\raggedleft\arraybackslash}p{#1}}
		\resizebox{\textwidth}{!}{%		
			\begin{tabular} { p{0.5cm} p{4.9cm} p{1.5cm} p{1.5cm} R{2.3cm}}
				\lsptoprule
				
				& &  \textbf{{Corpus }}& \textbf{{Experiment}}& \textbf{Overall}\\
				
				\midrule
				
				\textit{un}-	& Doubles longer than singletons& yes &  yes&  \multirow{2}{2.5cm}{\raggedleft Gemination}\\
				& Durational difference & large & very large&\\
				\midrule
				\textit{in}-		& Doubles longer than singletons& yes & mostly, stress-dependent &  \multirow{3}{2.5cm}{\raggedleft Gemination}\\
				&Durational difference & large &small&\\
				\midrule
				\textit{dis}-		& Doubles longer than singletons& mostly & mostly&  \multirow{2}{2.5cm}{\raggedleft Gemination}\\
				&Durational difference & small &small&\\
				\midrule
				-\textit{ly} &Doubles longer than singletons & no &   mostly not  & \multirow{2}{2.5cm}{\raggedleft Degemination}\\
				& Durational difference & none &small&\\					
				\lspbottomrule
			\end{tabular}
		} 
 
\end{table*}


By investigating \isi{gemination} with the different affixes, the predictions of various theories of the morpho-phonological and the \isi{morpho-phonetic interface} were tested. On the one hand, I tested predictions of formal linguistic theories. On the other, I tested predictions which were deduced from psycholinguistic approaches of the \is{morpho-phonetic interface}\is{morpho-phonological interface}morpho-phonological-phonetic interface. The psycholinguistic predictions center around two factors: \isi{decomposability} and \is{informativeness}morphological informativeness.  Below I will summarize and discuss the main findings of this book.



\section{Decomposability}

Decomposability is one of the main factors whose effect on \isi{gemination}, and on the \isi{acoustic realization} of words in general, was tested in this project. On the one hand, categorical \isi{decomposability} in terms of the overall \isi{segmentability} of the affix was investigated. On the other, the effects of word-specific \is{decomposability measure}decomposability measures were tested. 
Before testing the effect of the overall \isi{segmentability} of an affix, it was necessary to find out how segmentable each affix is. This was done by first deriving \isi{segmentability} hierarchies from the theoretical literature, and by then validating these hierarchies  in the corpus and the experimental study. Below I will first give a summary of the \isi{segmentability} analyses of the affixes. Then, I will summarize the effects of \isi{decomposability} found in both studies.



\subsection{The segmentability of the affixes}
%\enlargethispage{\baselineskip}

In Chapter \ref{affixes}, two \isi{segmentability} hierarchies, which order the five investigated affixes in terms of their \isi{segmentability}, were derived from the theoretical literature. They are shown below in (\ref{Non-Semantic Hierarchy})  and (\ref{Semantic Hierarchy}). 


	\begin{exe}
		\ex \label{Non-Semantic Hierarchy} {Non-Semantic Segmentability }\\
		Hierarchy:\hspace*{3.5cm}  \prefix{un} > \is{-ly}\suffix{ly} > \{\prefix{dis}, \prefix{in}\textsubscript{\textsc{Neg}}\} >  \prefix{in}\textsubscript{\textsc{Loc}}
		
				\ex \label{Semantic Hierarchy} 	{Semantic Segmentability }\\
				Hierarchy: \hspace*{3.5cm} \prefix{un} > \{\prefix{dis}, \prefix{in}\textsubscript{\textsc{Neg}}\} >  \prefix{in}\textsubscript{\textsc{Loc}} > \is{-ly}\suffix{ly}
	\end{exe}

The two hierarchies differ with regard to the definition of \isi{decomposability} they are based on. % in terms of their \isi{segmentability}. 
In the Non-Semantic Segmentability Hierarchy, \isi{decomposability} is defined in terms of \isi{productivity}, transparency and the type of base an affix takes. In the Semantic Segmentability Hierarchy, \isi{decomposability} is defined in terms of the semantics of the affix, i.e. an affix with an independent clear meaning is more segmentable than an affix without a clear meaning. Note that, as discussed in \sectref{morphological informativeness}, the Semantic Segmentability Hierarchy does not only capture the \isi{segmentability} of the affix but simultaneously represents its \isi{informativeness}. An affix with a clear lexical meaning is more informative than an affix without a clear meaning.


The comparison of the affix's \isi{segmentability} in both studies indicates that the two theoretically derived \isi{segmentability} hierarchies are borne out by the data.
 The corpus study showed the same \isi{segmentability} pattern across all \is{decomposability measure}decomposability measures.
 The prefix \is{un-}\prefix{un} and the suffix \is{-ly}\suffix{ly} are the most segmentable affixes, followed by \is{dis-}\prefix{dis} and \is{negative in-}negative \prefix{in}, and \is{locative in-}locative \prefix{in} is the least segmentable affix. This pattern matches the Non-Semantic Segmentability Hierarchy.


In the experimental study, the distribution of the decomposability ratings revealed a similar picture. The participants of the experimental study rated \is{un-}\prefix{un} as the most segmentable affix, \is{locative in-}locative \prefix{in} as the least segmentable affix, and \is{negative in-}negative \prefix{in} and \is{dis-}\prefix{dis} pattern in between. However, in contrast to the corpus study, the suffix \is{-ly}\suffix{ly} is rated as the second least segmentable affix after \is{locative in-}locative \prefix{in}. This pattern resembles the Semantic Segmentability Hierarchy in that \is{-ly}\suffix{ly} is one of the least segmentable affixes.

To sum up, the data shows that the five investigated affixes differ in their \isi{segmentability}. The prefix \is{un-}\prefix{un} is very segmentable and very informative, the prefixes \is{negative in-}negative \prefix{in} and \is{dis-}\prefix{dis} are less segmentable and informative, and \is{locative in-}locative \prefix{in} is the least segmentable and informative prefix. The \isi{segmentability} of the suffix \is{-ly}\suffix{ly} largely depends on the definition of \isi{decomposability}. While it is very segmentable in terms of its \isi{productivity}, its transparency and the type of base it takes, it is less segmentable in terms of its semantics. It is the least informative affix of the set of investigated affixes.
The two \isi{segmentability} hierarchies capture these findings. While the \isi{segmentability} pattern of the prefixes is the same in both hierarchies, the position of \is{-ly}\suffix{ly} differs between the two hierarchies. 


\subsection{Effects on the acoustic realization of words}

The studies revealed two types of \isi{decomposability} effects on {acoustic duration}: categorical \isi{segmentability} effects of the affix and gradient \is{decomposability}word-specific decomposability effects. Both types of effects go in the same direction: the more segmentable a unit is, the  longer it is, i.e. the less reduced it is. That there are both categorical and gradient effects of \isi{decomposability} is in line with former studies which also found different types of \isi{decomposability} effects (see, for example, \cite{Schuppler.2012}, or the discussion in \sectref{decomposability}). 

In the corpus study, the \isi{segmentability} of the affixes affected the duration of the nasal in \is{un-}\prefix{un} and \is{in-}\prefix{in}prefixed words. The most segmentable prefix \is{un-}\prefix{un} featured a longer nasal than the less segmentable prefix \is{negative in-}negative \prefix{in}, which in turn featured a longer nasal than the least segmentable affix \is{locative in-}locative \prefix{in}. 


In the experimental study, nasals in \is{un-}\prefix{un} and \is{in-}\prefix{in}prefixed words also differed in their duration. However, only double nasals were affected, i.e. the double nasal in \is{un-}\prefix{un}prefixed words was longer than the double nasal in \is{in-}\prefix{in}prefixed words. There was no clear difference between the duration of the double nasal in negative and \is{locative in-}locative \prefix{in}.

The experimental study, furthermore, showed effects of \is{decomposability}word-specific decomposability. The two \is{decomposability measure}decomposability measures \textsc{SemanticTransparencyRating} and log\textsc{RelativeFrequency} affected consonant duration with \is{-ly}\suffix{ly}. Items which were rated as less decomposable featured a shorter /l/ than items which were rated as more decomposable. Furthermore, for \is{-ly}\suffix{ly}-words with a \is{syllabicity}syllabic double consonant, derivatives with a higher \isi{relative frequency}, i.e. less decomposable items, featured a shorter /l/ than derivatives with a lower \isi{relative frequency}, i.e. more decomposable derivatives. However, as discussed in \sectref{ly experiment summary}, the effects of \is{decomposability}word-specific decomposability are very small and the effect of \isi{relative frequency} might be caused by a few items in the data set. 


To sum up, the present study shows that categorical \isi{segmentability} and \is{decomposability}word-specific decomposability may affect the {acoustic duration} of complex words.
 There are differences in the effect of categorical \isi{segmentability} between corpus and experimental study. This suggests
that \isi{segmentability} might be affected by \isi{speech mode}. 
Effects of \is{decomposability}word-specific decomposability were only found for the suffix \is{-ly}\suffix{ly}, not for the prefixes. This suggests that suffixes might be more affected by \is{decomposability}word-specific decomposability than prefixes.
 Crucially, the effects of \is{decomposability}word-specific decomposability do not interact with \isi{gemination}. 




\section{Morphological gemination: The overall picture}

Gemination is a categorical phenomenon. This is evidenced by the bimodal distribution of singletons and doubles found in the data. Furthermore, \is{morphological gemination}{gemination} is mostly governed by the affix, i.e. by a categorical factor. While some affixes geminate, others do not. In addition to the affix itself, for some affixes, the \isi{stress} pattern of a derivative affects \isi{gemination}. 
As shown in the experimental study, \is{morphological gemination}{gemination} does not depend on \isi{orthography}.

Even though \isi{gemination} is categorical in the sense that doubles are categorically longer than singletons, there are gradient differences in the \isi{degree of gemination} between affixes. The \isi{degree of gemination}, or the strength of \isi{gemination}, is mainly indicated by the durational differences between phonological doubles and phonological singletons. Stronger \isi{gemination} goes together with larger \is{singleton-geminate ratio}singleton-geminate ratios. 


The prefix \is{un-}\prefix{un} clearly geminates in both studies. In the corpus and the experimental study, phonological doubles (e.g. /nn/ in \textit{unnatural}) are clearly longer than singletons in complex words (e.g. /n/ in \textit{uneven} or \textit{untold}). The experimental study furthermore showed that doubles are longer than singletons in base words  (e.g. /n/ in \textit{natural}). Singleton-geminate ratios are bigger in the experimental than in the corpus study.

The prefix \is{in-}\prefix{in} also geminates but \isi{gemination} is weaker than \isi{gemination} with \is{un-}\prefix{un}. While \isi{gemination} with \is{in-}\prefix{in} is comparable to \isi{gemination} with \is{un-}\prefix{un} in the corpus study, in the experimental study \isi{gemination} with \is{in-}\prefix{in} is weaker and depends on \isi{stress}. 
 In the corpus study, 
 all doubles are longer than all singletons, and the durational differences between doubles and singletons are similar to the ones found for \is{un-}\prefix{un}. 
In the experimental study, 
the \is{singleton-geminate ratio}singleton-geminate ratios for \is{in-}\prefix{in} are smaller than the ones for \is{un-}\prefix{un}.
Furthermore, doubles are only longer than singletons when the base-initial syllable of a derivative is \is{stress}stressed, and doubles are only longer than some types of singletons. For /ɪn/, doubles (e.g. /nn/ in \textit{innumerous}) are longer than singletons in complex words followed by a vowel (e.g. /n/ in \textit{inefficient}), but not longer than singletons followed by a consonant (e.g. /n/ in \textit{intolerant}). For /ɪm/, doubles (e.g. /mm/ in \textit{immortal}) are longer than singletons in complex words followed by a consonant (e.g. /m/ in \textit{impossible}). Doubles in derived words are never longer than initial singletons in base words (e.g. /n/ in \textit{numerous} or /m/ in \textit{mortal}). 




The prefix \is{dis-}\prefix{dis} geminates in the corpus study as well as in the experimental study. In the corpus study, \isi{gemination} with \is{dis-}\prefix{dis} is weaker than \isi{gemination} with \is{un-}\prefix{un} and \is{in-}\prefix{in} in the sense that there is a smaller singleton-geminate ratio for \is{dis-}\prefix{dis} than for \is{un-}\prefix{un} and \is{in-}\prefix{in}. Furthermore, one \is{dis-}\prefix{dis}prefixed type did not geminate (\textit{dissolution}), presumably because of its unstressed base-initial syllable. 

In the experimental study, \isi{gemination} with \is{dis-}\prefix{dis} is weaker than \isi{gemination} with \is{un-}\prefix{un} but similar to \isi{gemination} with \is{in-}\prefix{in}.
All \is{dis-}\prefix{dis}prefixed words in the experimental data geminated. However, no morphological geminates with an unstressed base-initial syllable were included. 
The singleton-geminate ratio for \is{dis-}\prefix{dis} is smaller than that for \is{un-}\prefix{un}, and similar to that for \is{in-}\prefix{in}. 
Furthermore, in contrast to \is{un-}\prefix{un}, and similar to \is{in-}\prefix{in}, doubles are not longer than singletons in base words.  


For the suffix \is{-ly}\suffix{ly}, no \isi{gemination} was found. In the corpus study, double consonants with \is{-ly}\suffix{ly} were as long as singletons. In the experimental study, three different types of double consonants were investigated: \is{syllabicity}syllabic ones (e.g. /ll/ in  \textit{ment(a)lly}), non-\is{syllabicity}syllabic ones spelled with the orthographic sequence <lel> (e.g. /ll/ in \textit{solely}), and non-\is{syllabicity}syllabic ones spelled with <ll> (e.g. /ll/ in \textit{really}). While under certain conditions the \is{syllabicity}syllabic doubles and the doubles spelled with <lel> were longer than some types of singletons, there was no systematic difference between doubles and singletons, i.e. two underlying consonants are not longer than one. The suffix \is{-ly}\suffix{ly} degeminates.



Thus, the overall pattern of \is{morphological gemination}{gemination} is the following: the prefix \is{un-}\prefix{un} geminates. The prefixes \is{in-}\prefix{in} and  \is{dis-}\prefix{dis} also geminate but \isi{gemination} is weaker. For \is{in-}\prefix{in}, \isi{gemination} is only weaker than \isi{gemination} with \is{un-}\prefix{un} under experimental conditions. For \is{dis-}\prefix{dis}, \isi{gemination} is weaker in both the corpus and the experimental study. The suffix \is{-ly}\suffix{ly} degeminates. The overall \isi{gemination} pattern of the affixes is shown in (\ref{gemination hierarchy}) in form of a hierarchy.


\begin{exe}
	
	\ex \label{gemination hierarchy} {Gemination Hierarchy}: \hspace*{0.5cm}	{\prefix{un} > \{\prefix{in}\textsubscript{\textsc{Neg}} ,  \prefix{in}\textsubscript{\textsc{Loc}}\}> \prefix{dis} > \is{-ly}\suffix{ly}}
	
\end{exe}


A comparison with former empirical studies reveals that the results are largely in line with previous research. 
Apart from the studies presented in this book, three studies looked at affixational \is{morphological gemination}{gemination} in English, \cite{Kaye.2005}, \cite{Oh.2012} and \cite{Kotzor.2016} (see Chapter \ref{previous empirical work} for a detailed discussion). 
\cite{Kaye.2005} and \cite{Oh.2012} looked at \is{morphological gemination}{gemination} with \is{un-}\prefix{un} and \is{in-}\prefix{in}. Their results that both prefixes geminate fit in well with the results presented in this book.

\cite{Oh.2012} furthermore showed that the singleton-geminate ratio is smaller for \is{in-}\prefix{in} than for \is{un-}\prefix{un}, i.e. \is{in-}\prefix{in} geminates to a lesser degree than \is{un-}\prefix{un}, and that the difference in \is{singleton-geminate ratio}singleton-geminate ratios between \is{un-}\prefix{un} and \is{in-}\prefix{in} is even larger in \isi{careful speech} than in \isi{normal speech}. 
This fits in well with this study's result that there are only differences in the \isi{degree of gemination} between \is{un-}\prefix{un} and \is{in-}\prefix{in} in the experimental data, which presumably is more similar to \isi{careful speech} than the corpus data.

\cite{Kotzor.2016} looked at \isi{gemination} with \suffix{ness} and \is{-ly}\suffix{ly}. They claim that both suffixes geminate. Their result that \is{-ly}\suffix{ly} geminates is not in line with the results presented in this book. However, as thoroughly discussed in Chapter \ref{previous empirical work}, \cite{Kotzor.2016} do not provide separate analyses for the two affixes \suffix{ness} and \is{-ly}\suffix{ly}. It is thus questionable how valid their result is. 


\section{Corpus study vs. experimental study}


There is a peculiar difference between the affixes with regard to their behavior in the corpus study vs. their behavior in the experimental study.
While for some affixes, there is a difference in the \isi{degree of gemination} between the corpus and the experimental study, for others no such difference was observed. For the prefix \is{un-}\prefix{un}, \isi{gemination} is stronger in the experimental study than in the corpus study. For \is{in-}\prefix{in}, the opposite is the case: \isi{gemination} is weaker in the experimental study than in the corpus study. For \is{dis-}\prefix{dis} and \is{-ly}\suffix{ly}, no difference was found in the degree of (de)\isi{gemination} between the corpus study and the experimental study.

Before attempting to interpret the differences between the corpus results and the experimental results, it is important to note that all differences between the two studies are merely observed. In other words, no statistical analysis which directly compares the corpus and the experimental data was conducted. The reason for not conducting such an analysis is methodological in nature: the corpus data and the experimental data are too different to be analyzed in one statistical model. 
One major difference between the data sets is their size. The experimental data set features more than 13 times more observations than the corpus study. In a \is{regression model}{linear model}, this would lead to serious problems, as the model's estimates would be largely based on the experimental data.
 Furthermore, the compositions of the data sets in the two studies are only partly comparable, i.e. the data sets feature items of slightly different environments (see also \sectref{General Method Data Sets} for discussion). The fact that different variables were coded in the two studies also poses a problem for a direct comparison (see also \sectref{General method annotation} for discussion).

Returning to the observed differences, one might speculate that the different behavior of the affixes with regard to their \isi{gemination} pattern in the corpus and the experimental study is related to their \isi{segmentability}. It might, for example, be that highly segmentable affixes like \is{un-}\prefix{un} display stronger \isi{gemination} under experimental conditions, while less segmentable affixes like \is{in-}\prefix{in} show weaker \isi{gemination} under experimental conditions. However, this idea does not carry through. As described above, the prefix \is{dis-}\prefix{dis} is approximately as segmentable as \is{negative in-}negative \prefix{in}, and presumably more segmentable than \is{locative in-}locative \prefix{in}. If the differences between corpus and experimental study were related to \isi{segmentability}, \is{dis-}\prefix{dis} should behave as both \is{in-}\prefix{in}prefixes. It does not. %However, \is{dis-}\prefix{dis} behaves differently than \prefix{in}. 
It is thus highly questionable, whether the different behavior of the affixes with regard to their \isi{gemination} in the corpus and the experimental study is related to their \isi{segmentability}.

To conclude, there are differences between the results of the corpus and the experimental study but it is yet unclear how these differences can be explained. In order to find out what causes deviating results between corpus and experimental studies, it is necessary to better understand the differences between speech production in reading tasks and speech production in natural \isi{conversational speech}. In order to do so, more research combining corpus and experimental studies is necessary (see also \sectref{corpus and experimental studies}, or \cite{Arppe.2007} for discussion). 





\section{Implications for theory}


To theoretically interpret the results of the studies, we need to reconsider the approaches to the \isi{morpho-phonological interface} discussed in Chapter \ref{Theory}.   By comparing the actual \isi{gemination} pattern (as found in the studies) with the predictions these approaches make, we can find out which approach is supported by the data, and which approach is not.



\tabref{tbl:Factors predicting gemination 2} gives an overview of the discussed theoretical approaches. For each approach, the table shows the underlying concepts the approach assumes to govern \is{morphological gemination}{gemination}, and the variables which, according to the approach, are predicted to affect \isi{gemination} in the studies. For example, according to \isi{Lexical Phonology}, the \is{lexical stratum}stratum of the affix is decisive for its \isi{gemination}. In turn, the variable affix is predicted to affect \isi{gemination} in the studies. 
Variables which affected \isi{gemination} in the studies are printed in black, variables which did not affect \isi{gemination} are printed in gray.






\begin{table*} 
	\caption{Summary of underlying concepts and variables predicting gemination according to different theoretical approaches}
	\label{tbl:Factors predicting gemination 2} 
				\resizebox{\textwidth}{!}{%		
		\begin{tabular}{lll}
			\lsptoprule
			\textbf{Approach }& \textbf{Concept }& \textbf{Variable(s)} \\ 
			\midrule
			
			& \\
			
			\isi{Lexical Phonology} 									& \is{lexical stratum}stratum of affix &affix\\ 
			& \\
			\is{Stratal Optimality Theory}Stratal OT 													& \is{lexical stratum}stratum of affix& affix\\ 
			&type of base for & \color{lsLightGray}{type of base}\\ 
			& \is{dual-level affix}dual-level affixes&			\\ 
			& \\
			Prosodic Word										 & \isi{prosodic word} status  & affix \\ 
			& 								of affix						&  \color{lsLightGray}{semantic transparency}\\ 
			& 														& \color{lsLightGray}{type of base} \\ 
			& \\
			Morphological Segment- &\isi{decomposability} of  &  \color{lsLightGray}{relative frequency} \\ 
			ability (word-specific)															& 				derivative										& \color{lsLightGray}{semantic transparency} \\
			&														& \color{lsLightGray}{type of base} \\
			&														&\color{lsLightGray} {\isi{decomposability} rating}\\
			&														& \color{lsLightGray}{semantic similarity}\\
			
			& \\															
			Morphological Segment- &\isi{segmentability} &   affix\\	
			ability (affix-specific) &of affix	 & \\																	
			& \\
			Morphological Informative-& word-specific  & \color{lsLightGray}{semantic transparency}\\
			ness (word-specific) &\isi{informativeness}& \\																	 
			& \\
			Morphological Informative- & affix-specific  &  affix \\
			ness (affix-specific) &\isi{informativeness}& \\																	 
			
			
			\lspbottomrule
			
		\end{tabular}
	} 
\end{table*}



The studies revealed that \isi{gemination} in English is categorical and affix-specific. There are no word-specific effects on \isi{gemination}.\footnote{The only exception might be the word \textit{dissolution} which shows a particularly short fricative duration. As discussed thoroughly in \sectref{discussion experiment}, the analyses do not reveal what causes the short /s/ in \textit{dissolution}, but one plausible explanation is its unstressed base-initial syllable. In other words, it is assumed that its \isi{degemination} is not caused by type-specific factors but by more general prosodic factors.} Out of all the factors predicted to govern \isi{gemination}, only one actually affected \isi{gemination} in the studies, i.e. affix (see \tabref{tbl:Factors predicting gemination 2}). 
In addition to the affix, the variable \textsc{BaseInitialStress}, which is not predicted to govern \isi{gemination},  affected \isi{gemination} with the prefix \is{in-}\prefix{in} in the experimental study. 

Only those approaches which predict \isi{gemination} to be affix-dependent can be supported by the results. Word-specific approaches are not supported by the data.
Thus, the two word-specific approaches, i.e. word-specific Morphological Segmentability and word-specific Morphological Informativeness, are  not supported by the data. 

To find out which of the five approaches that assume affix-specific \isi{gemination} is supported by the data, we need to take a closer look at their predictions. Even though all five approaches expect  \isi{gemination} with some affixes and \isi{degemination} with others, they differ with regard to the \isi{gemination} pattern they predict. For example, while the three formal approaches are rather strict with regard to the expected \isi{gemination} or \isi{degemination} of a  certain affix, the two affix-specific psycholinguistic approaches predict an implicational \isi{gemination} pattern, i.e. they predict \isi{gemination} to follow a certain hierarchy. For example, if in a given hierarchy A<B<C  affix A and affix C geminate, affix B is also expected to geminate. 
We need to look at each approach individually to find out whether its predictions are supported or falsified by the data. 

%\isi{Lexical Phonology}
According to \isi{Lexical Phonology}, the \is{lexical stratum}stratum of the affix is decisive for \isi{gemination}. The \is{level 1 affix}level 1 affixes \is{in-}\prefix{in} and \is{dis-}\prefix{dis} are predicted to degeminate, and the \is{level 2 affix}level 2 affixes \is{un-}\prefix{un} and \is{-ly}\suffix{ly} are predicted to geminate. Except for the fact that the \is{level 2 affix}level 2 affix \is{un-}\prefix{un} geminates, all other predictions are wrong. The \is{level 1 affix}level 1 affixes \is{in-}\prefix{in} and \is{dis-}\prefix{dis} geminate, and the \is{level 2 affix}level 2 affix \is{-ly}\suffix{ly} degeminates. Thus, the stratal prediction is clearly falsified by the data.

% \is{Stratal Optimality Theory}Stratal OT
The predictions made by \is{Stratal Optimality Theory}Stratal OT are very similar to the predictions made by \isi{Lexical Phonology} as they are also based on lexical \is{lexical stratum}strata. 
The \is{level 2 affix}level 2 affixes \is{un-}\prefix{un} and \is{-ly}\suffix{ly} are predicted to geminate. The \isi{gemination} of the \is{dual-level affix}dual-level affixes \is{in-}\prefix{in} and \is{dis-}\prefix{dis} is predicted to depend on the type of base found in a pertinent word. Items with a bound root are predicted to degeminate, and items with words as bases are predicted to geminate. The \isi{degemination} of \is{-ly}\suffix{ly}, and the fact that \isi{gemination} of \is{in-}\prefix{in} and \is{dis-}\prefix{dis} is independent from a derivative's type of base, falsify the predictions made by \is{Stratal Optimality Theory}Stratal OT.  

%Prosodic Word
The Prosodic Word Approach predicts that all affixes forming independent \is{prosodic word}prosodic words geminate. All affixes not forming independent \is{prosodic word}prosodic words are predicted to degeminate. According to the approach, \is{un-}\prefix{un} always forms an independent \isi{prosodic word} and is thus predicted to always geminate, \is{-ly}\suffix{ly} never forms an independent \isi{prosodic word} and is thus predicted to never geminate. Gemination with \is{in-}\prefix{in} and \is{dis-}\prefix{dis} is predicted to depend on the derivative the prefix is found in, i.e. on the \isi{prosodic word} status of the prefix in the  derivative. 
According to \cite{Raffelsiefen.1999}, the \isi{prosodic word} status of prefixes is indicated by various features, such as the \isi{semantic transparency} of the derivative, its type of base and its \isi{stress} pattern. In this study, the two features \isi{semantic transparency} and type of base were used as indicators of \isi{prosodic word} status. All other listed features were not applicable (see discussion in \sectref{prosodic word}). 

The Prosodic Word Approach  makes the correct predictions for \is{un-}\prefix{un} and \is{-ly}\suffix{ly}. It also correctly predicts that \isi{gemination} with \is{in-}\prefix{in} and \is{dis-}\prefix{dis} is not as consistent as \isi{gemination} with \is{un-}\prefix{un}. However, the approach fails to predict the correct \isi{gemination} pattern for \is{in-}\prefix{in} and \is{dis-}\prefix{dis}. 
 Neither the \isi{semantic transparency} of a derivative nor its type of base significantly influences \isi{gemination} with \is{in-}\prefix{in} and \is{dis-}\prefix{dis}.
One can thus state, if \isi{prosodic word} status is defined by a derivative's \isi{semantic transparency} and its type of base, the Prosodic Word Approach is not supported by the data.


However, there is some evidence 
for the influence of prosodic structure on \isi{gemination}. 
Gemination with \is{in-}\prefix{in} in the experimental study depends on \isi{stress}. Only derivatives with a \is{stress}stressed base-initial syllable geminate. Furthermore, \isi{gemination} with \is{dis-}\prefix{dis} might also depend on \isi{stress}. As discussed in section  \ref{discussion experiment}, the \isi{degemination} of the word \textit{dissolution} is probably caused by its unstressed base-initial syllable, i.e. there is some evidence that \is{dis-}\prefix{dis} only geminates when the base-initial syllable of a derivative is \is{stress}stressed. 

Based on these results, one can speculate that using different criteria to determine \isi{prosodic word} status, such as \isi{stress}, might have led to better predictions of the Prosodic Word Approach for \is{in-}\prefix{in} and \is{dis-}\prefix{dis}. 
However, using \isi{stress} to determine \isi{prosodic word} status is very problematic. % and was therefore desisted from in this study.
 According to \cite{Raffelsiefen.1999}, \is{stress}prefixal stress is the crucial determiner for \isi{prosodic word} status, i.e. not base-initial \isi{stress} (see \sectref{prosodic word}). Prefixal \isi{stress} is, however, very difficult to determine and highly debated in the literature (see discussion on \is{stress}prefixal stress in Sections~\ref{affixes} and \ref{stress coding}). Therefore, it was not directly investigated in this study. Instead the effect of base-initial \isi{stress} on \isi{gemination} was tested, but the exact relation of base-initial \isi{stress} and \is{stress}prefixal stress is yet unclear.
Further studies which investigate the relation of base-initial \isi{stress}, \is{stress}prefixal stress and \isi{prosodic word} status are needed to clarify the matter. Only if valid criteria for \isi{prosodic word} status are available, one can further investigate whether \is{morphological gemination}{gemination} is governed by \isi{prosodic word} status. 

 




% Psycholinguistic Approaches
Let us now turn to the affix-specific psycholinguistic approaches. Both of them are based on the two \isi{segmentability} hierarchies postulated in Chapter \ref{affixes} (see  (\ref{Non-Semantic Hierarchy})  and (\ref{Semantic Hierarchy}) for the two hierarchies).
The two hierarchies differ in how they rank the lexical semantics of the affix in relation to \isi{productivity}, \isi{semantic transparency} and type of base. The Segmentability Approach is based on both hierarchies and does not specify the role of semantics for \isi{segmentability}, while the Morphological Informativeness Approach is solely based on the Semantic Segmentability Hierarchy.% According to the affix-specific psycholinguistic approaches, \isi{gemination} should pattern according to the hierarchies. 

Gemination does not pattern according to the Non-Semantic Segmentability Hierarchy. According to that hierarchy, \is{-ly}\suffix{ly} is more segmentable than \is{in-}\prefix{in} and \is{dis-}\prefix{dis}. As \is{in-}\prefix{in} and \is{dis-}\prefix{dis} geminate, one should also find \isi{gemination} with \is{-ly}\suffix{ly}. This is not the case. The suffix \is{-ly}\suffix{ly} does not geminate.  In turn, there is no support for affix-specific psycholinguistic approaches which predict \is{morphological gemination}{gemination} to pattern according to the Non-Semantic Segmentability Hierarchy.


According to the Semantic Segmentability Hierarchy, \is{-ly}\suffix{ly} is the least segment-able and the least informative affix. It is thus most likely to degeminate. This assumption is supported by the data, i.e. we find \isi{degemination} with \is{-ly}\suffix{ly}. There is also a difference in \isi{segmentability} and \isi{informativeness} between the four prefixes: \is{un-}\prefix{un} should geminate to a higher degree than \is{negative in-}negative \prefix{in} and \is{dis-}\prefix{dis}, which in turn should geminate to a higher degree than \is{locative in-}locative \prefix{in}. As shown in (\ref{gemination hierarchy}), this pattern is at least partly observed, \is{un-}\prefix{un} geminates to a higher degree than both \is{in-}\prefix{in}prefixes and \is{dis-}\prefix{dis}. There is, however, no difference in the \isi{degree of gemination} between \is{locative in-}locative \prefix{in} and \is{negative in-}negative \prefix{in}, and \isi{gemination} with \is{dis-}\prefix{dis} is a little weaker than \isi{gemination} with \is{in-}\prefix{in}. 


Overall though, the \isi{gemination} pattern supports the affix-specific psycholinguistic approaches which predict \isi{gemination} to pattern according to the Semantic Segmentability Hierarchy. The most segmentable and most informative affix \is{un-}\prefix{un} geminates, the least segmentable and informative affix \is{-ly}\suffix{ly} degeminates, and the other affixes pattern in between.
Thus, the affix-specific Segmentability Approach (if based on the Semantic Hierarchy) and the affix-specific Morphological Informativeness Approach are supported by the data. The more informative and segmentable an affix is in terms of its semantics, the higher is its \isi{degree of gemination}.


% Further evidence
Turning away from \isi{gemination}, the studies also revealed that \isi{decomposability} affects the \isi{acoustic realization} of complex words.
 In the corpus data, prefixal consonant durations for \is{un-}\prefix{un} and \is{in-}\prefix{in} reflect the \isi{segmentability} of the affix. 
The experimental study revealed gradient effects of \isi{decomposability} for \is{-ly}\suffix{ly}-suffixed words. 
Both findings go in the same direction: the more decomposable a unit is, the less \isi{reduction} is found. 

While these findings support dual route models of lexical access, in which \isi{decomposability} affects whether a complex word is accessed as a whole or via its parts (see, for example, \citealt{Frauenfelder.1992}; \citealt{Schreuder.2015}; \citealt{deVaan.2011}; \citealt{Caselli.2016})
, there are still questions which remain unanswered. The studies found effects of categorical affix-specific \isi{segmentability}, as well as \is{decomposability}word-specific decomposability effects. It remains unclear how these two different types of \isi{decomposability} effects interact. Furthermore, while there are effects of \is{decomposability}word-specific decomposability for \is{-ly}\suffix{ly}, there are no effects for the prefixes. This evokes the question whether there is a difference between the retrieval and the processing of suffixed words on the one hand, and prefixed words on the other. Further studies are needed to investigate this difference and shed light on the interplay between categorical and word-specific effects of \isi{decomposability}. Only then can adequate {models of word storage and retrieval} be specified. 



% Speech Processing
For theories of speech production, the result that \is{locative in-}locative \prefix{in} and \is{negative in-}negative \prefix{in} differ in the duration of their nasal in the corpus study has important implications. 
As a number of other studies before, this outcome shows that morphological structure is mirrored in \isi{phonetic detail}, i.e. that  acoustic realizations are not solely based on \is{phonemic representation}phonemic representations (see \sectref{speech production models} for an overview of studies on the topic). Currently, \is{speech production model}{models of speech production} are unspecified with regard to the \is{morphological processing}{processing of complex words} (see, for example, \citealt{Dell.1986,Johnson.1997b,Levelt.1999b,Bybee.2002,Pierrehumbert.2001,Pierrehumbert.2002}). They must be revised, or specified, in order to account for the fact that the \isi{acoustic realization} of complex words shows traces of morphological structure.

