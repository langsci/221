\chapter {Morphological gemination: Implications for theory} \label{Theory}

% Outlook of chapter
In this chapter, I will show how the {gemination} pattern of English affixes can shed light on various theories of the morpho-phonological and \isi{morpho-phonetic interface}. I will discuss three different fields of theory which directly or indirectly predict the \isi{gemination} pattern of English: Formal linguistic theories, psycholinguistic approaches of \isi{morphological processing} and theories of speech production. After giving a general overview, I will discuss each field  separately, explain its main branches and deduce the predictions made for \is{morphological gemination}{gemination} with the five affixes investigated in this study.  At the end of the chapter, I will summarize the predictions of all theories discussed.\\


\section{Overview}
One can distinguish two major branches of morpho-phonological (or morpho-phonetic) theories. The first set of theories is formal, categorical and often generative in nature. Generally, in this type of theory it is assumed that phonological entities are abstract. Furthermore, these theories do not allow for a direct \isi{morpho-phonetic interface}. Phonetic detail is thus believed to not mirror morphological structure directly. Depending on the theory, \is{phonological rule}phonological rules and prosodic structure can, however, indirectly mirror the morphological structure of complex words and lead to phonetic effects. 
In these formal theories, \isi{morphological gemination} is regarded as a categorical morpho-phonological and rule-based process, i.e. not as a gradient, word-specific process of the direct \isi{morpho-phonetic interface}. Degemination is categorical and occurs when, due to some \isi{phonological rule} or process, one segment of the double consonant is deleted. This deletion is then mirrored in phonetics. Depending on the theory, the predictions of which structures undergo \isi{degemination}, and which structures display \isi{gemination}, deviate. I will look at three prominent formal linguistic approaches in this book, i.e. \isi{Lexical Phonology}, newer \is{stratal approach}stratal approaches such as \isi{Stratal Optimality Theory}, and \isi{Prosodic Phonology},  and discuss their predictions for \isi{gemination} with the five affixes looked at in this study. 

% Second branch of theories#

The second branch of theories is psycholinguistic in nature and incorporates gradient, probabilistic factors. Experience-based effects, such as probabilities and \is{frequency}frequencies of occurrence, play a major role in this type of theory. In contrast to formal approaches, theories of this kind do not assume \isi{gemination} to be categorical and rule-based, but rather to be gradient and word-specific. In  general these psycholinguistic approaches assume a more direct relation between morphological structure and phonetics than formal theories by, for example, assuming that fine \isi{phonetic detail} may be stored in the \isi{mental lexicon}. In this study, I will concentrate on two prominent factors in psycholinguistic theory: \isi{decomposability} and \is{informativeness}morphological informativeness. These two factors are believed to influence the \isi{phonetic realization} of complex words, and should thus also play a prominent role in \isi{morphological gemination}. 

% Distinction is not absolute

The binary classification of theories into formal, categorical approaches and gradient, probabilistic approaches must not be regarded as absolute. Formal linguistic theories may incorporate gradient, probabilistic factors, and in psycholinguistic approaches categorical concepts may be found. However, even though the categorization is not absolute, it helps to get a systematic overview of approaches which propose explanations for morpho-phonological and morpho-phonetic variation. The categorization simplifies the comparison of similar approaches, and helps with the identification of changes and developments of theories.

%Speech Production Theories
In addition to the two types of theories outlined above, I will also discuss speech production theories. There are various ways of how the morpho-phono-logical (or the morpho-phonetic) interface is modeled in speech production.  
I will present two different types of models. In the first type, a strict feed-forward structure with no explicit \isi{morpho-phonetic interface} is assumed. The second type of model is based on exemplars, i.e. concrete mental entities which are formed from experience. While the former fits in better with the categorical approaches outlined above, the latter can be regarded to be more in the spirit of gradient approaches. 
However, I will desist from drawing explicit relations between \is{speech production model}speech production models and the different formal linguistic and psycholinguistic approaches discussed. The reason is that these relations are never explicitly mentioned in the literature, and that, due to the underspecification of morpho-phonological aspects in \is{speech production model}speech production models, a lot of information is not available that would be needed to relate the different theories with each other.
I will therefore review the two \is{speech production model}speech production models on their own. I will discuss them with respect to their understanding of morpho-phonological processes (such as, for example, {morphological gemination}), and will especially focus on the already mentioned underspecified aspects of the \is{morphological processing}processing of complex words.  Because of these underspecifications no explicit predictions for {gemination} in English {affixation} will be drawn from these models. However, I will discuss more general assumptions the models make with regard to the \isi{phonetic realization} of complex words. Some of these assumption can and will be tested in this study.



\section{Formal linguistic theories} 

In this section, I will discuss the following formal linguistic approaches: \isi{Lexical Phonology}, newer \is{stratal approach}stratal approaches such as \isi{Stratal Optimality Theory}, and \isi{Prosodic Phonology}. Within the framework of \isi{Prosodic Phonology}, I will concentrate on the \is{prosodic word}\newterm{prosodic word}.  In all three approaches morpho-phonological processes are assumed to be categorical in nature. Furthermore, in all of them affixes are believed to play a crucial role in morpho-phonological processes. The three approaches assume \isi{gemination} to be a categorical process, i.e.\ either the two consonants of a \is{morphological gemination}morphological geminate remain and we find \isi{gemination}, or one consonant is deleted and we find \isi{degemination}. In all of the approaches \isi{boundary strength} is one of the most important determiners for phonological behavior, i.e. also for \isi{gemination}. 
One can thus see that there are quite a few similarities in the approaches. However, there are also important differences between them. One of these differences is the type of boundary assumed to be decisive for the phonological behavior of a derivative. While \isi{Lexical Phonology}, for example, assumes affix-specific lexical differences in \isi{boundary strength}, \isi{Prosodic Phonology} defines \isi{boundary strength} mainly in phonological terms. 
Another difference between the approaches is the assumed role of gradiency in morpho-phonological processes.
Crucially, these differences between approaches lead to different predictions for \isi{gemination}. In the following, I will discuss each approach and present the predictions it makes for \isi{gemination} with the five affixes investigated in this study.


\subsection{Lexical Phonology} \label{LexPhon}


\isi{Lexical Phonology} is one of the first theories which discussed the morpho-phono-logical interface and which makes clear predictions about {gemination} in English. According to \isi{Lexical Phonology} (cf. \citealt{Kiparsky.1982,Mohanan.1986}), all morphological processes are carried out by modules in the lexicon. These modules are called levels or \is{lexical stratum}strata. A complex word is formed by retrieving entries from the lexicon which then undergo the morpho-phonological processes of the different \is{lexical stratum}strata. The \is{lexical stratum}strata are linked to specific affixes and are connected to \is{phonological rule}phonological rules, which are carried out at the level they belong to. At each level one or more affixes can be added to a word. After affixation, the \is{phonological rule}phonological rules adjust the form of the derivative so that the phonotactics of the language the word is formed in are met. During the word formation process an item can pass each level several times. A word does not leave a \is{lexical stratum}stratum until all operations the \is{lexical stratum}stratum holds for the word are carried out, i.e. all derivational or inflectional processes of a level are completed. This concept is called \newterm{cyclicity}. After a word has passed all \is{lexical stratum}strata, it leaves the lexicon and undergoes post-lexical rules. These rules adjust the word's sub-phonemic and word-external features. 
At the \isi{post-lexical stage} the complex word does not carry any morphological information.  This is due to the principle of \newterm{bracket erasure}, which erases all morphological boundaries after each lexical level. Hence, according to \isi{Lexical Phonology} the phonetics of a complex word does not show traces of its morphological structure.

\figref{fig:Word Formation Process imperfectness} depicts the word formation process for the complex word \textit{imperfectness}, which consists of the \is{level 1 affix}level 1 prefix \is{in-}\prefix{in}, the root \textit{perfect} and the \is{level 2 affix}level 2 suffix \textit{-ness}.\footnote{Note that the figure only displays two levels, i.e. \is{lexical stratum}level 2 and \is{lexical stratum}level 2. These are the two levels relevant for affixation in English, i.e. the two levels relevant for this study. The number of assumed levels differs between different variants of \isi{Lexical Phonology} (see \citealt{Giegerich.1999} for an overview), the level 1- level 2-distinction, as displayed in \figref{fig:Word Formation Process imperfectness}, is, however, agreed on by all variants.}
First, the root \textit{perfect} is retrieved from the lexicon and passed to the first level. After the \is{level 1 affix}level 1 affix \is{in-}\prefix{in} is added to the root, \is{lexical stratum}level 2 morphology passes the concatenated form to \is{lexical stratum}level 2 phonology, which applies the pertinent lexical rules. In this case, the adjacency of the two phonemes /n/ and /p/ triggers the application of an assimilation rule which changes /n/ to /m/. After this change in form, the word is passed back to \is{lexical stratum}level 2 morphology to check whether further affixes need to be added. Since this is not the case, the word leaves \is{lexical stratum}level 2 and is sent to the second level. 
Before entering \is{lexical stratum}level 2, all morphological boundaries of the derivative are erased (due to \isi{bracket erasure}). When \textit{imperfect} undergoes \is{lexical stratum}level 2 processes it is thus treated the same way as a simplex word.  
In \is{lexical stratum}level 2, \is{lexical stratum}level 2 morphology adds the suffix \textit{-ness} to the derivative. The attachment of \textit{-ness} does not change the phonological form of the word, i.e. no \is{phonological rule}phonological rules are applied. The derivative goes through \is{lexical stratum}level 2 morphology and phonology again to check whether additional morphemes need to be added. It then leaves the lexicon, i.e. it is passed to the \isi{post-lexical stage}.\\

\begin{figure*}
	
	\centering
	\resizebox{.95\linewidth}{!}{%
	\begin{tikzpicture}[scale=0.7,every node/.style={scale=0.8}, framed]
	
	
	\node (LexicalEntries) [LexicalEntries] {Lexical entries};
	
	\node (MorphPhon1) [MorphPhon, below=2cm of LexicalEntries,  text width= 3 cm] {Morphology \textit{in + perfect imperfect}};
	\node (MorphPhon2) [MorphPhon, below=6cm of LexicalEntries,  text width= 3cm] {Morphology \\ \textit{imperfect + ness  imperfectness}};
	%	\node (MorphPhon3) [MorphPhon, below=10cm of LexicalEntries,  text width= 3 cm] {Morphology \textit{imperfectness}};
	\node (MorphPhon4) [MorphPhon, below=2cm of LexicalEntries,right =3cm of MorphPhon1,  text width= 6.7 cm ] {Phonology\\ \textit{in + perfect}   $\rightarrow$  /ɪmpɜɾfəkt/   \\  /ɪmpɜɾfəkt/ };
	\node (MorphPhon5) [MorphPhon, below=6cm of LexicalEntries,right =3cm of MorphPhon2,  text width= 6.7cm] {Phonology\\ \textit{imperfect + ness} $\rightarrow$  /ɪmpɜɾfəktnəs/ \\  /ɪmpɜɾfəktnəs/};
	%\node (MorphPhon6) [MorphPhon, below=10cm of LexicalEntries,right =3cm of MorphPhon3,  text width= 6.7cm ] {Phonology \\  /ɪmpɜɾfəktnəs/ };
	
	\node (PostLex) [PostLex, below=9.5cm of LexicalEntries, xshift=4.5cm  ] {Post-lexical rule application};
	
	
	% need to position the labels level
	
	\node (Level2) [Level, below=5.5 cm of LexicalEntries, xshift=4.5cm] {};
	%\node (Level3) [Level, below=9.5cm of LexicalEntries, xshift=4.5cm ] {};
	\node (Level1) [Level, below= 1.5 cm of LexicalEntries, xshift= 4.5 cm] {};
	
	\node (Level2caption) [below= 5.7  cm of LexicalEntries, xshift = -3cm] {Level 2};
	%\node (Level3caption) [below= 9.7  cm of LexicalEntries, xshift = -3cm] {Level 3};
	\node (Level1caption) [below= 1.7  cm of LexicalEntries, xshift = -3cm] {Level 1};
	
	
	\draw [arrowright] (LexicalEntries) -| node[midway,right, text width= 2 cm]{Retrieval of the root \textit{perfect}}(Level1);
	\draw [arrowright] (MorphPhon5) -- node [left, xshift=-1.5cm,  text width= 1.5 cm] {Bracket Erasure} (PostLex);
	
	%	\draw [arrowright] (MorphPhon6) -- node[left, xshift=-1.5cm,   text width= 1.5 cm] {Bracket Erasure} (PostLex);
	
	\draw [arrow2] (MorphPhon4.north west) --(MorphPhon1.north east);
	\draw [arrow2] (MorphPhon1.east) --(MorphPhon4.west);
	\draw [arrow2] (MorphPhon4.south west) --(MorphPhon1.south east);
	
	\draw [arrow2] (MorphPhon5.north west) --(MorphPhon2.north east);
	\draw [arrow2] (MorphPhon2.east) --(MorphPhon5.west);
	\draw [arrow2] (MorphPhon5.south west) --(MorphPhon2.south east);
	
	%	\draw [arrow2] (MorphPhon3) --(PostLex);
	
	\draw [arrowright] (MorphPhon4) -- node[left, xshift=-1.5cm,  text width= 1.5 cm] {Bracket Erasure} (MorphPhon2);	
	\end{tikzpicture}}
	\caption{Word formation process of \textit{imperfectness} in Lexical Phonology\label{fig:Word Formation Process imperfectness}}
\end{figure*}


As suggested by this example, \is{level 1 affix}level 1 affixes differ from \is{level 2 affix}level 2 affixes with regard to their \isi{boundary strength}. The notion of \isi{boundary strength}, as applied here, goes back to \cite{Chomsky.1968}. They assumed two different types of morphological boundaries: strong ones and weak ones.  While \is{level 1 affix}level 1 affixes trigger weak morphological boundaries, which go along with a high degree of phonological integration, \is{level 2 affix}level 2 affixes form strong boundaries and integrate less. The example \textit{imperfectness}  illustrates the difference. While the adding of the \is{level 1 affix}level 1 affix \is{in-}\prefix{in} leads to assimilation, neither the phonological  form of the base, nor the one of the suffix changes when the \is{level 2 affix}level 2 affix \textit{-ness} is added. One should, however, note that, even though the notion of \isi{boundary strength} corresponds well with the theory of \isi{Lexical Phonology}, \citet[239]{Kiparsky.1985} explicitly states that by introducing the different \is{lexical stratum}strata the notion of \isi{boundary strength} becomes unnecessary in \isi{Lexical Phonology}. 

Transferring the concept of lexical \is{lexical stratum}strata and its implications for the phonological relation between affix and base to {gemination}, the following picture emerges: a \is{morphological gemination}morphological geminate will degeminate when part of a \is{level 1 affix}level 1 affix, and it will geminate when part of a \is{level 2 affix}level 2 affix (see \citealt[18]{Mohanan.1986}). To find out what is predicted for the affixes under investigation, we need to determine which level each affix belongs to. 
Based on phonological and lexical properties, as well as on affix origin, \isi{Lexical Phonology} assumes that the affixes \is{dis-}\prefix{dis} and \is{in-}\prefix{in} belong to \is{lexical stratum}level 2, and that the affixes \is{un-}\prefix{un} and \is{-ly}\suffix{ly} belong to \is{level 2 affix}level 2. Note that no differentiation between locative and negative \is{in-}\prefix{in} is made. Both belong to \is{lexical stratum}level 2.
Comparing the distinction between \is{level 1 affix}level 1 and \is{level 2 affix}level 2 affixes with the \isi{segmentability} hierarchies proposed in the previous chapter, one can see that the distinction resembles the Non-Semantic Segmentability Hierarchy (\is{un-}\prefix{un} > \is{-ly}\suffix{ly} > \{\is{dis-}\prefix{dis}, \is{in-}\prefix{in}\textsubscript{\textsc{Neg}}\} >  \is{in-}\prefix{in}\textsubscript{\textsc{Loc}}). This hierarchy is based on lexical factors and ranks \isi{productivity} and transparency above lexical meaning.  


\figref{fig:LP predictions} depicts \isi{Lexical Phonology}'s prediction for \isi{gemination} (and \isi{degemination}) with the five affixes. Due to a morphological process a \is{morphological gemination}morphological geminate, i.e. a double consonant, emerges. For the \is{level 1 affix}level 1 affixes negative \is{in-}\prefix{in}, locative \is{in-}\prefix{in} and \is{dis-}\prefix{dis} the theory predicts the deletion of one of the two adjacent consonants by some kind of phonological process. This deletion is expected to lead to a short duration of the \is{morphological gemination}morphological geminate, i.e. to \isi{degemination}. For the affixes \is{un-}\prefix{un} and \is{-ly}\suffix{ly} no deletion is expected, i.e. no phonological process is applied. The double consonant remains and the \is{morphological gemination}morphological geminate is therefore realized with a long duration. The affixes \is{un-}\prefix{un} and \is{-ly}\suffix{ly} geminate.  
To summarize, according to \isi{Lexical Phonology}, the \is{level 2 affix}level 2 affixes \is{un-}\prefix{un} and \is{-ly}\suffix{ly} are expected to geminate, and the \is{level 1 affix}level 1 affixes negative \is{in-}\prefix{in}, locative \is{in-}\prefix{in} and \is{dis-}\prefix{dis} are expected to degeminate. 

%\newpage
As described above, \isi{Lexical Phonology} assumes \isi{gemination}, i.e. the duration of the double consonant, to be mostly influenced by the type of affix involved. In other words, the affix involved affects consonant duration the most. However, there are additional factors which are assumed to also influence duration, namely post-lexical factors (e.g. \isi{speech rate}, the preceding segment). Importantly, these factors are expected to be found on top of the expected effect of the affix. Lexical factors which mirror an individual word's morphological structure are not expected to play a role. All traces of morphological structure are erased by \isi{bracket erasure}. 
The predictions of \isi{Lexical Phonology} are summarized in \figref{fig:LP predictions}.



\begin{figure*}  
	
	\centering
		\resizebox{.95\linewidth}{!}{%
	\begin{tikzpicture}[scale=0.8,every node/.style={scale=0.8}, framed]

	
	\node (Morph) [LingLevel,xshift=-8cm] {Morphologiocal Process};
	\node (Phono) [LingLevel, below=1 cm of Morph ] {Phonological Process};
	\node (Phone) [LingLevel, below= 1 cm of Phono] {Phonetic Realization};

	\node (MorphLev1) [LingLevel,right= 1.5cm of Morph] {\textit{in} + \textit{numerous} \\ \textit{in} + \textit{migrate}\\ \textit{dis} + \textit{satisfied}};
	\node (PhonoLev1) [LingLevel, right=1.5cm of Phono] {\textit{i}/n/\textit{umerous} \\ \textit{i}/m/\textit{igrate} \\ \textit{di}/s/\textit{atisfied}};
		
	\node (PhoneLev1) [LingLevel, right=1.5cm of Phone]{\textit{i}[n]\textit{umerous} \\ \textit{i}[m]\textit{igrate} \\ \textit{di}[s]\textit{atisfied}};

	\node (MorphLev2) [LingLevel,right=1.5 cm of MorphLev1] {\textit{un} + \textit{natural} \\ \textit{total} + \textit{ly}};
	\node (PhonoLev2) [LingLevel, right=1.5 cm of PhonoLev1] {\textit{u}/nn/\textit{atural} \\ \textit{tota}/ll/\textit{y}};
	\node (PhoneLev2) [LingLevel, right=1.5 cm of PhoneLev1]{\textit{u}[n:]\textit{atural} \\ \textit{tota}[l:]\textit{y}};
	
	\node(Level 1) [LingLevel, above= 1cm of MorphLev1]{\textbf{Level 1}};
	\node(Level 2) [LingLevel, above= 1cm of MorphLev2]{\textbf{Level 2}};
	
	\node(degemination) [LingLevel, below= 1cm of PhoneLev1]{\textbf{degemination}};
	\node(gemination) [LingLevel, below= 1cm of PhoneLev2]{\textbf{gemination}};	
	


	\draw [arrow4] (MorphLev1) --(PhonoLev1);
	\draw [arrow4] (PhonoLev1) --(PhoneLev1);
	\draw [arrow4] (PhoneLev1) --(degemination);
	\draw [arrow4] (MorphLev2) --(PhonoLev2);
	\draw [arrow4] (PhonoLev2) --(PhoneLev2);
	\draw [arrow4] (PhoneLev2) --(gemination);
		
	\end{tikzpicture}}
	\caption{Lexical Phonology: predictions for gemination with \prefix{un}, negative \prefix{in}, locative \prefix{in}, \prefix{dis} and\suffix{ly}}
	 \label{fig:LP predictions}
\end{figure*}




\newpage\largerpage
\begin{description}\item[\isi{Lexical Phonology}: \is{morphological gemination}Predictions]
\begin{enumerate}[leftmargin=*]
	\item[]
	\item \prefix{un} geminates \\
	The nasal in \prefix{un}prefixed words with a phonological single consonant will be shorter than the nasal in \prefix{un}prefixed words with a phonological double consonant. %\\
	%-- \hspace{0.15cm} The /n/ in \textit{uneven} will be shorter than the /n/ in \textit{unnatural}.
	
	\item \prefix{in} degeminates \\
	The nasal in \prefix{in}prefixed words with a phonological single consonant will be as long as the nasal in  \prefix{in}prefixed words with a phonological double consonant.%\\ 
	%-- \hspace{0.15cm} The /n/ in \textit{inevident} will be of the same length as the /n/ in \textit{innumerous}. \\
	%-- \hspace{0.15cm} The /m/ in \textit{implant} will be of the same length as the /n/ in \textit{immigrational}. 
	
	
	
	\item \prefix{dis} degeminates \\
	The fricative in \prefix{dis}prefixed words with a phonological single consonant will be as long as the fricative in \prefix{dis}prefixed words with a phonological double consonant. %\\ 
	%-- \hspace{0.15cm} The /s/ in \textit{disarm} will be of the same length as the /s/ in \textit{dissatisfy}. 
	
	
	
	\item \suffix{ly} geminates \\
	The lateral in \suffix{ly}-suffixed words with a phonological single consonant will be shorter than the lateral in \suffix{ly}-suffixed words with a phonological double consonant.%\vspace{0.15cm}\\ 
	%-- \hspace{0.15cm} The /l/ in \textit{truly} will shorter than the /l/ in \textit{really}. 
	
	\item Post-lexical factors affect consonant duration.
	
	\item Factors which mirror the morphological structure of individual words do not affect consonant duration.
	
\end{enumerate}
\end{description}

\subsection{Newer stratal approaches}

During the last decades various problems with \isi{Lexical Phonology} have been revealed. These problems are closely related to the strict feed-forward structure of the model, as well as the strict division of labor between the \is{lexical stratum}strata (see also \citealt[Chapter 2]{Giegerich.1999} and \citealt[Chapter 7]{Plag.2003} for a thorough discussion). The criticism of \isi{Lexical Phonology} has led to modifications of the theory and the development of newer \is{stratal approach}stratal approaches, which are argued to solve the problems found with the original approach. Two of these approaches will be discussed in this section: the base-driven approach proposed by \cite{Giegerich.1999} and \isi{Stratal Optimality Theory}. To understand the changes made in these newer approaches, it is necessary to take a closer look at the problems with the original approach.

One of the most prominent problems with \isi{Lexical Phonology} is affix ordering. According to \isi{Lexical Phonology}, \is{level 1 affix}level 1 affixes are attached to a base before \is{level 2 affix}level 2 affixes, and \is{level 2 affix}level 2 affixes are attached before inflectional affixes.\footnote{According to most varieties of Lexical Phonology (e.g.  \citealt{Kiparsky.1982,Mohanan.1986}) inflectional affixes belong to an additional \is{lexical stratum}stratum which follows derivation.} Therefore, \is{level 2 affix}level 2 affixes are not expected to be found within \is{level 1 affix}level 1 affixes, and inflectional affixes are not expected to be found within \is{lexical stratum}level 2 or \is{level 2 affix}level 2 affixes. However, in some derivatives these unexpected structures are found. For example, in the derivative \textit{interestingly} the inflectional suffix \suffix{ing} precedes the \is{level 2 affix}level 2 suffix \is{-ly}\suffix{ly},\footnote{Note that this structure does not pose a problem to \isi{Lexical Phonology} if \is{-ly}\suffix{ly} is regarded as an inflectional suffix. However, \isi{Lexical Phonology} adapts the view that \is{-ly}\suffix{ly} is derivational. } and in the derivative \textit{ungrammaticality} the \is{level 1 affix}level 1 affix \suffix{ity} is attached after the \is{level 2 affix}level 2 affix \is{un-}\prefix{un}. Examples like these call into question whether the strict feed-forward structure proposed by \isi{Lexical Phonology} is valid (see also \citealt[Chapter 4]{Plag.1999} for discussion).

Another problem with \isi{Lexical Phonology} is the variation found within affixes. According to the theory, all affixes of one \is{lexical stratum}stratum, and thus all derivatives of one affix, should display the same (or at least very similar) behavior with regard to morpho-phonological processes. However, recent literature has pointed out that there is variation within the \is{lexical stratum}strata and even within the derivatives of one affix (cf., for example, \citealt{Raffelsiefen.1999,Bauer.2013, Plag.2014, BermudezOtero.2017}).  One prominent example of this variation is \isi{stress shift}. The suffix \textit{-able}, for instance, is expected to preserve \isi{stress}. There are, however derivatives with \suffix{able} in which \isi{stress shift} is possible (e.g. \textit{ˈanalyze -- ˈanalyzable \textasciitilde\ anaˈlyzable}). With some derivatives \isi{stress shift} even seems to  be consistent (e.g. \textit{ˈcategorize -- categoˈrizable}) (cf. \citealt[213f.]{Plag.2014}). Empirical studies on \isi{stress} confirm variation within affixes, i.e. different derivatives of the same affix deviate in their \isi{stress} pattern (see for example \citealt{Collie.2008} for \suffix{ion} and \suffix{ity}, \citealt{Sanz.2017} for \suffix{ory}). The categorization of affixes as either \is{level 1 affix}level 1 or \is{level 2 affix}level 2 is insufficient to explain the variation found. 

A related, and maybe more essential, problem is the categorization of an affix as either \is{level 1 affix}level 1 or \is{level 2 affix}level 2. As summarized by \citet[134]{Raffelsiefen.1999}, \is{level 1 affix}level 1 affixes feature bound roots, are unproductive, determine \isi{stress}, trigger assimilation and yield idiosyncratic meaning. Level 2 affixes attach to words, are productive, are stress-neutral, block assimilation and yield compositional meaning. However, as can be seen in \tabref{tbl:The affixes under investigation: a comparison} in \sectref{comparison affixes}, affixes often simultaneously feature \is{level 1 affix}level 1 and \is{lexical stratum}level 2 properties. Negative \is{in-}\prefix{in}, for example, assimilates, i.e. features a \is{lexical stratum}level 2 property, and is productive and \is{semantic transparency}semantically transparent, i.e. simultaneously features \is{lexical stratum}level 2 properties. This shows that the assignment of an affix to one or the other level is often not clear-cut. 

The three problems just discussed, i.e. affix ordering, intra-affix variation and the categorization of affixes, demonstrate that the strict division between \is{level 1 affix}level 1 and \is{level 2 affix}level 2 affixes, as well as the strict sequential order of \is{lexical stratum}level 2 and \is{lexical stratum}level 2 processes, are insufficient to explain the variation found in English derivatives. While this insufficiency has led some linguists to  abandon the idea of lexical \is{lexical stratum}strata in general (cf., for example, \citealt{Johnson.1997b,Bybee.2001,Pierrehumbert.2001,Hay.2001}), others have modified the \isi{stratal approach}. 
\cite{Giegerich.1999}, for example, proposes a base-driven \isi{stratal approach}. For English, he assumes two lexical \is{lexical stratum}strata: a root \is{lexical stratum}stratum and a word \is{lexical stratum}stratum. In the first \is{lexical stratum}stratum, i.e. the root \is{lexical stratum}stratum, all structure-changing morpho-phonological operations are carried out. In the second \is{lexical stratum}stratum, i.e. the word \is{lexical stratum}stratum, all structure-building processes take place. The important difference between Giegerich's approach and \isi{Lexical Phonology} is that in Giegerich's approach, it is not the affix which is decisive for the \is{lexical stratum}stratum a derivative is formed in. Instead, the distinction between \is{lexical stratum}stratum 1 and \is{lexical stratum}stratum 2 is based on a derivative's base. If the base is a word, the derivative is formed in \is{lexical stratum}stratum 2. If the base is a root, the derivative is formed in \is{lexical stratum}stratum 1. In light of categorical approaches of morpho-phonology, \isi{degemination} can be classified as a structure-changing process, i.e. a \is{lexical stratum}stratum 1 process. Thus, only derivatives featuring a root, i.e. \is{lexical stratum}stratum 1 derivatives, are expected to display \isi{degemination}. Derivatives with words as their base are expected to geminate.

There are two major problems with Giegerich's approach  and the predictions it makes for \is{morphological gemination}{gemination} in English affixation. The first problem is that prefixation is not discussed, and that the approach is thus restricted to suffixation. One could assume that the predictions for suffixes can be extended to prefixes, i.e. one could predict prefixed words with words as bases to geminate and prefixed words with roots as bases to degeminate. It is, however, unclear whether this extension would comply with the approach. 
The second problem concerns the categorization of bases as roots or words. Giegerich does not give clear criteria for this categorization. This makes the testing of predictions for \is{morphological gemination}{gemination} which are based on the distinction of word vs. root almost impossible. 
Furthermore, Giegerich states speaker-dependent differences, i.e. the same base might be a root for one speaker and a word for another (cf. \citealt[Chapter 3.2.1.]{Giegerich.1999}). 
Due to these assumed differences among speakers, post-hoc explanations are available for all cases. If a word geminates, the speaker can be argued to have stored the base as a word. If a word degeminates, the opposite can be argued, i.e. the speaker has stored the base as a root. Because of these post-hoc explanations, the base-driven approach suggested by Giegerich might not be falsifiable. It is therefore not tested in this study.

\isi{Stratal Optimality Theory} (Stratal OT) is another, newer development of \is{stratal approach}stratal approaches. It combines the modular feed-forward structure of \isi{Lexical Phonology} with OT mechanisms (cf. \citealt{BermudezOtero.2012,BermudezOtero.2013,Kiparsky.2015,BermudezOtero.2017}). As in classical \isi{Lexical Phonology}, the grammar is organized into three levels: two lexical levels and one \isi{post-lexical level}. The first lexical level is the stem-level, the second is the word-level. The categorization of an affix as \is{level 1 affix}level 1 or 2 largely depends on the type of base it attaches to. If an affix attaches to a stem, it belongs to \is{lexical stratum}level 2, if it attaches to a word, it belongs to \is{level 2 affix}level 2 (cf. \citealt[7]{Kiparsky.2015}; \citealt[9f.]{BermudezOtero.2017}).  Thus, as in Giegerich's approach, the base of a derivative plays a crucial role in \is{Stratal Optimality Theory}Stratal OT. Also similar to Giegerich's approach, the categorization of a base as a stem or a word is not always clear. However, it can be assumed that all bound roots are classified as stems. Free-standing words might be classified as words. Crucially, in contrast to Giegerich's approach, speaker-dependent variation is not discussed in \is{Stratal Optimality Theory}Stratal OT, i.e. post-hoc classifications are not an issue.

\is{Stratal Optimality Theory}Stratal OT assumes each \is{lexical stratum}stratum to have its own constraint system. This system is responsible for phonological operations, i.e. also for \isi{degemination}. As noted by \citet[5]{Kiparsky.2015}, the constraint system at each level consists solely of input/output and markedness constraints. No additional constraints are assumed. The order of constraints may deviate between the levels. For morphological geminates one can assume a different ranking of constraints at each of the two lexical levels. This difference leads to \isi{degemination} on the stem-level and \isi{gemination} on the word-level. It is thus assumed that \is{level 1 affix}level 1 affixes degeminate, while \is{level 2 affix}level 2 affixes geminate. This prediction is very similar to the prediction made by \isi{Lexical Phonology}. There is, however, an important difference between the predictions of the two approaches. This difference is rooted in the assumption of \is{dual-level affix}dual-level affixes.

Different from classical \is{stratal approach}stratal approaches, \is{Stratal Optimality Theory}Stratal OT allows for \is{dual-level affix}dual-level affixes. According to \is{Stratal Optimality Theory}Stratal OT, these affixes behave like \is{level 1 affix}level 1 affixes when attached to a stem and like \is{level 2 affix}level 2 affixes when attached to a word (cf. \citealt[15, 33]{BermudezOtero.2017}). 
\citet[33]{BermudezOtero.2017} notes the prefix \is{in-}\prefix{in} to be one of these affixes. He states that while in derivatives like \textit{importune} \is{in-}\prefix{in} behaves like a stem-level prefix, in derivatives like \textit{impolite} it behaves like a word-level prefix. The comparison of the five affixes investigated in this study revealed that the prefix \is{dis-}\prefix{dis} behaves similarly to the prefix \is{in-}\prefix{in} (cf. \sectref{comparison affixes}). Like \is{in-}\prefix{in}, \is{dis-}\prefix{dis} also features \is{lexical stratum}level 2 and \is{lexical stratum}level 2 properties, e.g. it attaches to words and to bound roots. It can therefore be claimed that \is{dis-}\prefix{dis} also is a \isi{dual-level affix}.  


%\enlargethispage{2\baselineskip}
Since the behavior of \is{dual-level affix}dual-level affixes depends on the type of base they attach to, variation in the \isi{gemination} of \is{in-}\prefix{in} and \is{dis-}\prefix{dis} is expected. The prefixes are expected to geminate when attached to a word. They are expected to degeminate when attached to a stem. As discussed above, the distinction between stem and word is, however, not always clear. In turn, the predictions are not always clear. However, for some words clear predictions can be formed. Since all bound roots are classified as stems, \is{in-}\prefix{in} and \is{dis-}\prefix{dis} are expected to degeminate in all derivatives with a bound root. Only in derivatives with a word as a base \is{in-}\prefix{in} and \is{dis-}\prefix{dis} can geminate.
The predictions for \is{un-}\prefix{un} and \is{-ly}\suffix{ly} are very straightforward. According to \is{Stratal Optimality Theory}Stratal OT, the two affixes belong to the second \is{lexical stratum}stratum. The prediction for \is{un-}\prefix{un} and \is{-ly}\suffix{ly} therefore remains the same as in \isi{Lexical Phonology}. Both affixes are expected to geminate.



Apart from \is{dual-level affix}dual-level affixes, there is another important difference between \isi{Lexical Phonology} and \is{Stratal Optimality Theory}Stratal OT.  This difference is related to the {lexical storage} of complex words. While according to \isi{Lexical Phonology} all complex words are computed online, \is{Stratal Optimality Theory}Stratal OT assumes stem-level derivatives to be stored as a whole (cf. \citealt[Chapter 3]{BermudezOtero.2012}). Therefore, according to \is{Stratal Optimality Theory}Stratal OT, \isi{frequency} effects on \is{lexical stratum}level 2 derivatives are expected.  Words with a high \isi{frequency} are expected to be reduced phonetically, i.e. should also display shorter durations of the affixational consonant. This \isi{reduction} process is, however, not expected for \is{lexical stratum}level 2 derivatives which are only stored analytically (cf. \citealt[Chapter 3.3]{BermudezOtero.2012}). 



One can summarize that while some of the basic assumptions of \isi{Lexical Phonology} remained in newer approaches (e.g. modular feed-forward structure, lexical vs. \isi{post-lexical level}), there are also some important differences (e.g. \is{dual-level affix}dual-level affixes, {whole-word storage}). These differences lead to different predictions for \isi{gemination} with the five affixes of this study. While the prediction for the \is{level 2 affix}level 2 affixes \is{un-}\prefix{un} and \is{-ly}\suffix{ly} remains the same, i.e. they are expected to geminate, variability is expected for the other two affixes. Another important difference to \isi{Lexical Phonology} is that \is{Stratal Optimality Theory}Stratal OT includes psycholinguistic, gradient factors, such as \is{frequency}frequencies, by assuming {whole-word storage} of stem-level derivatives. Below the predictions of \is{Stratal Optimality Theory}Stratal OT are summarized.


\begin{description}\item[Stratal Optimality Theory: \is{morphological gemination}Predictions]
\begin{enumerate}[leftmargin=*]
	\item[]
	\item \is{un-}\prefix{un} geminates \\
	The nasal in \is{un-}\prefix{un}prefixed words with a phonological single consonant will be shorter than the nasal in  \is{un-}\prefix{un}prefixed words with a phonological double consonant. 
	%The /n/ in \textit{uneven} will be shorter than the /n/ in \textit{unnatural}.
		
	\item\is{in-}\prefix{in} degeminates in derivatives with a bound root\\
	 The nasal in \is{in-}\prefix{in}prefixed words with a phonological single consonant will be as long as the nasal in  \is{in-}\prefix{in}prefixed words with a phonological double consonant if  the derivative has a bound root as its base.
	%The /n/ in \textit{in} will be of the same length as the /n/ in \textit{inject}. \\
	%The /m/ in \textit{im} will be of the same length as the /n/ in \textit{im}. 

	\item \is{in-}\prefix{in} geminates in derivatives with a word as a base \\
	The nasal in \is{in-}\prefix{in}prefixed words with a phonological single consonant is shorter than the nasal in \is{in-}\prefix{in}prefixed words with a phonological double consonant if  the derivative has a word as its base.
	%The /n/ in \textit{inevident} will be shorter than the /n/ in \textit{innumerous}. \\
	%The /m/ in \textit{implant} will be shorter than the /n/ in \textit{immigrational}. 



	
	\item \is{dis-}\prefix{dis} degeminates in derivatives with a bound root\\
	The fricative in \is{dis-}\prefix{dis}prefixed words with a phonological single consonant will be as long as the fricative in \is{dis-}\prefix{dis}prefixed words with a phonological double consonant if  the derivative has a bound root as its base.
	%The /s/ in \textit{dis} will be of the same length as the /s/ in \textit{diss}. 

	\item \is{dis-}\prefix{dis} geminates in derivatives with a word as a base \\
	The fricative in \is{dis-}\prefix{dis}prefixed words with a phonological single consonant is shorter than the fricative in \is{dis-}\prefix{dis}prefixed words with a phonological double consonant if  the derivative has a word as its base.
	%The /s/ in \textit{disarm} will be shorter than s the /s/ in \textit{dissatisfy}. 

		
	\item  \is{-ly}\suffix{ly} geminates \\
	The lateral in \is{-ly}\suffix{ly}-suffixed words with a phonological single consonant will be shorter than the lateral in  \is{-ly}\suffix{ly}-suffixed words with a phonological double consonant. 
	%The /l/ in \textit{truly} will shorter than the /l/ in \textit{really}. 
	
	\item Post-lexical factors affect consonant duration.
	


	\item Word \isi{frequency} influences the duration of {level 1} derivatives, i.e. \is{in-}\prefix{in} and \is{dis-}\prefix{dis}prefixed words with a bound root.
	
\end{enumerate}
\end{description}


\subsection{The prosodic word} {\label{prosodic word}}


%Empirical work: morhological structure has indeed an effect on phonetis detail
Empirical work has shown that the morphological structure of a derivative may be directly mirrored in its \isi{acoustic realization}. %(see for example \citealt{Sproat.1993b,Cho.2001,Sugahara.2009,Pluymaekers.2010,Schuppler.2012,Smith.2012,LeeKim.2013,Blazej.2015,Plag.2017}). 
\cite{Sproat.1993b}, for example, found differences in the \isi{phonetic realization} of /l/ depending on the \is{boundary strength}strength of the morphological boundary the sound occurred at. Stronger morphological boundaries were found to feature a darker and longer /l/ than weaker morphological boundaries. Similarly, \cite{LeeKim.2013} found that the \isi{phonetic realization} of /l/ depends on the morphological structure of the derivative. 
For Dutch \is{homophone}homophones, \cite{Schuppler.2012} found more \isi{reduction} with complex words than with simplex words, i.e. word-final /t/ was more often deleted in simplex than in complex words.

With regard to duration, \cite{Cho.2001,Sugahara.2009,Hanique.2011,Smith.2012} and \cite{Plag.2017} found systematic differences between monomorphemic and morphemic words with similar phonological structures. For Korean, \cite{Cho.2001} found articulatory evidence on the variability of intergestural timing in monomorphemic and complex words. In an EPG study, the timing of the gestures for [ti] and [ni] shows more variation when the sequence is heteromorphemic (i.e. across a morpheme-boundary) than when it is tautomorphemic (i.e. without straddling a boundary).
For English, \citet{Sugahara.2009} found phonetic differences between the final segments of a monomorphemic stem as against the final segments of the same stem if followed by a suffix. Stems followed by \is{level 2 affix}level 2 suffixes had slightly longer rhymes than their monomorphemic counterparts. \citet{Smith.2012} discovered systematic phonetic differences in the realization of the first three segments between prefixed words and what they call \textit{pseudo-prefixed words} (such as \textit{mis-time} versus \textit{mistake}, respectively). Similarly \cite{Plag.2017} found a difference between morphemic and non-morphemic /s/ in English. They found non-morphemic /s/ to be longer than morphemic /s/. 


%no \isi{morpho-phonetic interface} for stratal theories
The studies above raise the question of how to model the relation between morphology and phonetics. Even though the studies yield somehow contradicting results, i.e. in some of the studies stronger morphological boundaries lead to more \isi{reduction} (e.g. \citealt{Sugahara.2009, Smith.2012}) and in some the opposite is the case (e.g. \citealt{Schuppler.2012, Plag.2017}), it seems certain that there is a stable effect of morphology on the \isi{acoustic realization} of a word. One might thus suggest a \isi{morpho-phonetic interface} which allows phonetics to directly access morphological information. This idea is picked up mainly by psycholinguistic approaches of \isi{morphological processing}, which completely abandon the strictly feed-forward structure of \is{stratal approach}stratal theories (cf. \sectref{Morphological Gemination: Implications for Psycholinguistic Theories of Morphological Processing}). 
In contrast, most formal theories hold on to the modular structure. These theories do not feature a direct \is{morpho-phonetic interface}{interface between morphology} and phonetics. Instead, they explain the effects of morphology on phonetics by referring to the prosodic structure of complex words (cf., for example, \citealt{Booij.1983b,Sproat.1993,Nespor.2007,Sugahara.2009,Bergmann.}). The prosodic structure of a derivative is closely connected to its morphological structure and is believed to directly influence the \isi{phonetic realization} of a word. 



% Prosodic boundaries
A very important concept in prosodic approaches, especially with regard to duration, is \isi{prosodic boundary} strength. The common assumption is that the stronger the \isi{prosodic boundary}, the less \isi{reduction} is found. The strength of a boundary in prosodic terms highly depends on the prosodic domain it is adjacent to.
\figref{fig:Prosodic Hierarchy} shows the \isi{prosodic hierarchy}, which depicts the different prosodic domains.\footnote{Note that the constituents of the hierarchy differ slightly depending on author and approach. The hierarchy displayed here is taken from \citet[9]{Hall.2001}.} 
The higher the domain in the hierarchy, the stronger the boundary and the less \isi{reduction} is expected. Segments followed by an intonational phrase boundary are, for example, expected to be less reduced than segments followed by a {phonological word} boundary.\footnote{The \isi{phonological word} is also called \is{prosodic word}\textit{prosodic word} or \is{p-word}{\textit{p-word}}. In this book, I will use the terms interchangeably.}
 The effects of \isi{prosodic boundary} strength are assumed to be additive, i.e. the more boundaries are present, the less \isi{reduction} is expected. 

\begin{figure}
        \begin{forest}
        [phonological utterance (U)
            [intonational phrase (IP)
                [phonological phrase ($\phi$)
                    [\isi{phonological word} ($\omega$)
                        [foot (F)
                            [syllable ($\sigma$)]]]]]]
        \end{forest}
		\caption{Prosodic hierarchy\label{fig:Prosodic Hierarchy}}
\end{figure}

% Phonological word: intro
Morphological geminates in English affixed words can occur at phonological word boundaries, foot boundaries and syllable boundaries. Since ``[t]he \isi{phonological word} ($\omega$) represents the interaction between the phonological and the morphological components of the grammar'' (\citealt[109]{Nespor.2007}), the phonological word boundary is the most important for this study. In other words, \isi{gemination} might be determined by phonological word boundaries, which mirror the morphological structure of a derivative and influence the \isi{acoustic realization} of the complex word. The \isi{phonological word} can be regarded as a mediator between the morphology and the phonetics of a complex word.

% garmmatical words and complex words
 A grammatical word can consist of one or more \is{prosodic word}prosodic words (cf., for example, \citealt[29]{Booij.1983b}; \citealt[267]{Booij.1985}; \citealt[2]{Hall.2001}). Importantly, only complex words can consist of several \is{prosodic word}prosodic words. This is due to the fact that \isi{prosodic word} boundaries must align with morphosyntactic boundaries (see for example \citealt[2]{Hall.2001}). Crucially, not every morphological boundary corresponds to a \isi{p-word} boundary. In other words, while some affixes form separate \is{prosodic word}prosodic words, and thus occur at a \isi{p-word} boundary, others form a \isi{prosodic word} together with their base, i.e. are not adjacent to a \isi{p-word} boundary. 
Examples of complex words with differing \isi{prosodic word} structures are given below. The affixes in the words in \REF{ex:4:1} constitute independent \is{prosodic word}prosodic words. The affixes in the words in \REF{ex:4:2} do not form independent p-words. They form one \isi{prosodic word} together with their base. 

\begin{exe}
	\ex\label{ex:4:1} (\textit{un})\textsubscript{$\omega$}(\textit{natural})\textsubscript{$\omega$}, (\textit{un})\textsubscript{$\omega$}(\textit{told})\textsubscript{$\omega$}
	\ex\label{ex:4:2} (\textit{really})\textsubscript{$\omega$}, (\textit{inject})\textsubscript{$\omega$}
\end{exe}\pagebreak


As stated by \citet[3]{Hall.2001}, \is{prosodic word}prosodic words can form the domain of phonological and prosodic rules. It can thus be assumed that \is{prosodic word}prosodic words also form the domain of \is{morphological gemination}{gemination}. While derivatives in which the affix forms a \isi{prosodic word} on its own (e.g. (\textit{un})\textsubscript{$\omega$} (\textit{natural})\textsubscript{$\omega$}) geminate, affixes which form a \isi{prosodic word} together with their base (e.g. \textit{(really)}\textsubscript{$\omega$}) degeminate (cf. \citealt[3543]{Giegerich.2012}; \citealt{Bergmann.}). In other words, \isi{reduction}, i.e. \isi{degemination}, is only found when the \is{morphological gemination}morphological geminate does not occur across a \isi{prosodic word} boundary. The crucial question now is how to determine whether a \is{morphological gemination}morphological geminate occurs across such a boundary, i.e. whether an affix constitutes a \isi{prosodic word} on its own.



It is generally assumed that \isi{prosodic word} structure corresponds to \is{boundary strength}morphological boundary strength. Stronger morphological boundaries form phonological word boundaries. However, the specific criteria for determining \is{prosodic word}prosodic words status are still debated in the literature (cf. \citealt{Raffelsiefen.1999}; \citealt{Hall.2001} for an overview).\footnote{Note that the criteria also deviate between languages. The overview given here is limited to the \isi{prosodic word} in English.} 
In earlier approaches, \isi{prosodic word} status was merely a mirror image of the level 1-\is{lexical stratum}level 2 distinction made in \is{stratal approach}{stratal theory}. While level-1 affixes were assumed to be integrated in the \isi{p-word} of their derivative, \is{level 2 affix}level 2 affixes were assumed to form a \isi{p-word} on their own (see, for example, \citealt{Aronoff.1983,Booij.1983b,Szpyra.1989}). According to this view, the predictions for \is{morphological gemination}{gemination} made by \isi{Prosodic Phonology} would be identical to the ones made by \isi{Lexical Phonology}. Level 2 affixes form an independent \isi{p-word} and thus geminate, \is{level 1 affix}level 1 affixes do not form an independent \isi{p-word} and thus degeminate. 

More recent approaches to the \isi{prosodic word} deviate from the stratal categorization. The \isi{prosodic word} status of English suffixes, for example, is debated fairly often in the literature. The general assumption is that suffixes do not form \is{prosodic word}prosodic words on their own (see, for example, \citealt[311]{Wennerstrom.1993}; \citealt[184]{Raffelsiefen.1999}; \citealt[401]{Hall.2001b}; \citealt{Sugahara.2009}). There are, however, different views with regard to the question of whether suffixes follow the \isi{prosodic word} formed by their base (e.g. \textit{(run)\textsubscript{$\omega$}\suffix{ing}}), or whether they are integrated in the \isi{prosodic word} of their base (e.g. \textit{(running)\textsubscript{$\omega$}}). While \cite{Sugahara.2009}, for example, suggest that all \is{level 2 affix}level 2 suffixes follow the \isi{p-word} formed by their base, \cite{Raffelsiefen.1999} and \cite{Hall.2001b} propose that this is only true for some of those suffixes. They suggest that only consonant-initial suffixes, such as \suffix{ness} and \is{-ly}\suffix{ly}, follow the \isi{prosodic word} which is formed by their base. Vowel-initial suffixes, such as \suffix{ing} or \suffix{er}, are integrated in the \isi{prosodic word} of their base. Examples of the \isi{p-word} structures according to the two different approaches are given below. As can be seen, the suffix \is{-ly}\suffix{ly}, which is under investigation in this study, is preceded by a \isi{p-word} boundary in both approaches. It is never integrated in the \isi{p-word} of its base.


\begin{multicols}{3}
	
	\begin{exe}
		
		\ex		
		\textbf{derivative}	\\
		\textit{run}\suffix{ing}	\\
		\textit{cool}\suffix{er} 	\\
		\textit{cool}\suffix{ness}\\
		\textit{cool}\suffix{ly} \\			
		\columnbreak 
		
		\textbf{Raffelsiefen }	\\
		\textit{(running)\textsubscript{$\omega$}} 	\\
		\textit{(cooler)\textsubscript{$\omega$}} \\
		\textit{(cool)\textsubscript{$\omega$}\suffix{ness}} \\
		\textit{(cool)\textsubscript{$\omega$}\suffix{ly}} \\
		\columnbreak
		
		\textbf{Sugahara \& Turk}	\\
		\textit{(run)\textsubscript{$\omega$}\suffix{ing}} \\
		\textit{(cool)\textsubscript{$\omega$}\suffix{er}} 	\\	
		\textit{(cool)\textsubscript{$\omega$}\suffix{ness}}\\
		\textit{(cool)\textsubscript{$\omega$}\suffix{ly}}	\\
		
	\end{exe}
	
\end{multicols}


%Wennerstr
Turning to the \isi{prosodic word} status of English prefixes, two approaches are to be discussed, \cite{Wennerstrom.1993} and \cite{Raffelsiefen.1999}. \cite{Wennerstrom.1993} suggests the distinction between analyzable and non-analyzable prefixes. Prefixes which are analyzable form \is{prosodic word}prosodic words, prefixes which are not analyzable do not. The key criterion for analyzability is focusability. If a prefix can be focused, it is analyzable and forms a \isi{prosodic word}. 
Importantly, \isi{prosodic word} status depends on individual derivatives and not on the prefix involved. This means that the same prefix might be considered a \isi{prosodic word} in one derivative but not in another (\citealt[314]{Wennerstrom.1993}). 
As an example, \citet[311]{Wennerstrom.1993} presents the word \textit{external}, in which the prefix \prefix{ex} can be focused in a sentence such as \textit{The country has both INternal and EXternal problems}. As shown in this example, analyzability is independent of whether the prefix's base is a bound root or a word. 

%Problems Wenners.
One problem with Wennerstrom's approach, as pointed out by \citet[161f.]{Raffelsiefen.1999}, is that focusability does not correlate with phonological characteristics of p-words. Furthermore, it seems that almost every prefix can be focused under certain conditions (see also \citealt[Chapter 4]{Plag.2003}). In normal, \isi{conversational speech} there might be the additional problem of determining whether a prefix is focused or not. For these reasons focusability is not considered a useful, reliable criterion for determining \isi{prosodic word} status.

%Raffelsiefen
\cite{Raffelsiefen.1999} proposes that \isi{segmentability} correlates with \isi{prosodic word} status. In more \is{decomposability}decomposable words, i.e. in derivatives in which the prefix is more segmentable, prefixes form independent \is{prosodic word}prosodic words. In less \is{decomposability}decomposable words, they do not.
She suggests that morphological structure, i.e. \isi{decomposability}, is translated into prosodic structure, which in turn is mirrored in the \isi{phonetic realization} of a derivative. Derivatives made up of only one \isi{prosodic word}, i.e. derivatives with less segmentable prefixes, are realized similarly to simplex words. The \isi{phonetic realization} of prefixes which form \is{prosodic word}prosodic words on their own, i.e. highly segmentable prefixes, is not affected by their base. 
A prefix's degree of \isi{segmentability} in a given derivative is defined by the semantic and phonological properties of the derivative. 
Semantic and phonological aspects of \isi{segmentability} are assumed to correlate with each other, an assumption which seems to be true for the affixes of this study (cf. \sectref{comparison affixes}). On top of \isi{decomposability}, \citet[175f.]{Raffelsiefen.1999} assumes an influence of \isi{frequency} on \isi{prosodic word} structure. 
She suggests that derivatives of very high \isi{frequency} are more likely to be parsed as a whole. In other words, high \isi{frequency} derivatives are processed as single \is{prosodic word}prosodic words irrespective of \isi{decomposability}.


\begin{table*}
	\caption{Criteria for prosodic word status of English prefixes \citep{Raffelsiefen.1999}}
	\label{tbl:Criteria for prosodic work status of English prefixes (Raffelsiefen 1999)}
	\begin{tabular}{lll}
		\lsptoprule
		{Criterion} & {Prosodic word }&{No \isi{prosodic word}}\\ 
		&{\textit{(prefix)\textsubscript{$\omega$}(base)\textsubscript{$\omega$}}}& {\textit{(prefix base)\textsubscript{$\omega$}}}\\
		\midrule 
		
		Stress & Secondary \isi{stress} on prefix & Primary \isi{stress} on prefix or  \\ 
		& 												& unstressed prefix \\
		&(e.g. \textit{ˌinˈtolerant, ˌdisˈhonor}) & (e.g. \textit{ˈimpotent, inˈdifferent})
		\\ 
		
		Syllabification & No syllabification of  & Syllabification of  \\
		&  prefixal coda 				& prefixal coda \\
		& (e.g. \textit{dis.integrate})& (e.g. \textit{di.sease}) 
		\\ 
		
		Aspiration & Aspiration of 		& No aspiration of  \\
		& base-initial stops &  base-initial stops \\
		& (e.g. \textit{dis[kʰ]olor}) & (e.g. \textit{dis[k]over})
		\\  
		
		
		Flapping & No flapping of  		& Flapping of\\
		&  base-initial stop 	& base-initial stop\\
		& (e.g. \textit{in[tʰ]olerant}) & (e.g. \textit{in[ɾ]egrate})
		\\   
		
		Type of derivative & Mostly of native origin & Mostly loanwords \\ 
		& (e.g. \textit{unpleasant, unjust}) & (e.g. \textit{inject, impotent})
		\\ 
		
		Meaning & Strictly compositional & Not compositional\\
		& (e.g. \textit{unpleasant, impolite}) & (e.g. \textit{inject, impotent})
		\\  
		
		Type of base & Word as a base & Bound root as a base\\
		& (e.g. \textit{unpleasant, impolite}) & (e.g. \textit{inject, innocent})
		\\
		\lspbottomrule                                                                                
	\end{tabular}
\end{table*}


To determine the \isi{segmentability} of an affix, i.e. \isi{prosodic word} status of a prefix, \cite{Raffelsiefen.1999} proposes a number of criteria. The criteria relevant for the prefixes under investigation are displayed in \tabref{tbl:Criteria for prosodic work status of English prefixes (Raffelsiefen 1999)}. Unfortunately, not all of the criteria turn out to be useful. Since {prefixal stress} is very hard to determine (especially in derivatives with base-initial primary \isi{stress}) the stress-criterion is not suited to reliably determine \isi{prosodic word} status (cf. \sectref{description un} for discussion of {prefixal stress}). Similarly, syllabification cannot always easily be determined and is thus not helpful. 
With regard to aspiration and flapping, these criteria are not applicable to all derivatives since they are restricted to specific sounds. In all derivatives with a double consonant a vowel follows the consonant, i.e. for none of these words the aspiration or the flapping criterion is applicable. Therefore, these two criteria cannot be used in this study.

The type-of-derivative-criterion merely displays a tendency and can therefore not be applied. 
This leaves us with two useful criteria: meaning and type of base. Prefixes which are \is{semantic transparency}{semantically transparent} and feature a word as a root are assumed to form independent \is{prosodic word}{prosodic words}. Prefixes which are semantically opaque and feature a bound root are assumed to form \is{prosodic word}{prosodic words} together with their base. 

There are two problems with the \isi{prosodic word approach} and its predictions for \isi{gemination}. 
The first problem is that the predictions made are restricted to certain combinations. No predictions are made for semantically opaque words with words as bases, or for \is{semantic transparency}{semantically transparent} words with bound roots. 
The second problem refers to the fact that only two criteria are applicable to determine \isi{prosodic word} status, meaning and type of base. This is problematic because both criteria are lexical, i.e. not prosodic. Neither the meaning of a word nor its type of base can serve as direct evidence for a word's prosodic structure, and one must acknowledge the possibility that possible effects of the two factors on \isi{gemination} are not due to \isi{prosodic word} structure but to other lexical mechanisms. 
Nonetheless, it is justified to apply the two criteria \textit{meaning} and \textit{type of base} to test \isi{prosodic word} status (as defined by Raffelsiefen). 

According to  \citeauthor{Raffelsiefen.1999}'s (1999) approach, meaning and type of base correlate with prosodic aspects of \isi{phonological word} structure, e.g. \isi{stress} and \isi{resyllabification}. Therefore, these two lexical criteria can be applied to determine \isi{prosodic word} status. 
One must, however, keep in mind that determining \isi{prosodic word} status by applying lexical measures of \isi{decomposability} is based on the \isi{prosodic word approach} by \cite{Raffelsiefen.1999}, i.e. by the assumption of a close connection between prosodic and lexical word structure. There is the possibility that effects of meaning and type of base are independent from effects of prosodic structure. Prosodic criteria, such as \isi{syllabicity} and \isi{stress}, would be more direct determiners of \isi{phonological word} structure. As described above, they are, however, not applicable, and further research on these prosodic criteria is needed to validly test \isi{phonological word} status. 


\begin{table*}
	\caption{Prosodic word statuses of \prefix{un}, \prefix{in}, \prefix{dis} and \suffix{ly}\label{tbl:Prosodic word statuses of affixes}}
	\begin{tabular}{llll}
		\lsptoprule
		Affix & \textit{(base)\textsubscript{$\omega$} suffix} & \textit{(prefix)\textsubscript{$\omega$}(base)\textsubscript{$\omega$}} & \textit{(prefix base)\textsubscript{$\omega$}}\\\midrule 
		\prefix{un}&& \textit{(un)\textsubscript{$\omega$}(natural)\textsubscript{$\omega$}}& \\ 
		\prefix{in}& &{(im)\textsubscript{$\omega$}(polite)\textsubscript{$\omega$}} &\textit{(inject)\textsubscript{$\omega$}} \\ 
		\prefix{dis}& &{(dis)\textsubscript{$\omega$}(trust)\textsubscript{$\omega$}}&  \textit{(dissipate)\textsubscript{$\omega$}}\\  
		\suffix{ly}& \textit{(cool)\textsubscript{$\omega$} }\suffix{ly}	 & &\\   
		\lspbottomrule                                                                                
	\end{tabular}
\end{table*}

\tabref{tbl:Prosodic word statuses of affixes} summarizes the analysis of the \isi{prosodic word} status of  \is{un-}\prefix{un}, \is{in-}\prefix{in}, \is{dis-}\prefix{dis} and \is{-ly}\suffix{ly}. The predictions for \is{morphological gemination}{gemination} made by the \isi{prosodic word approach} are based on this analysis. Affixes forming independent \is{prosodic word}prosodic words are expected to geminate, affixes which do not form independent \is{prosodic word}prosodic words are predicted to degeminate.
According to Raffelsiefen's approach, the three prefixes \is{un-}\prefix{un}, \is{in-}\prefix{in} and \is{dis-}\prefix{dis} form \is{prosodic word}prosodic words when they are part of a \is{semantic transparency}semantically transparent derivative with a word as a base. The prefixes are integrated in the \isi{prosodic word} of their base when they are part of a semantically opaque derivative with a bound root. 
Note that, as discussed in \sectref{description un}, derivatives with \is{un-}\prefix{un} are always \is{semantic transparency}semantically transparent and feature words as a base. The prefix \is{un-}\prefix{un} is thus expected to always form a \isi{prosodic word} and to always geminate. 
Gemination with \is{in-}\prefix{in} and \is{dis-}\prefix{dis} is predicted to depend on the type of base the prefix takes in a given derivative. For \is{semantic transparency}semantically transparent derivatives with words as bases, \isi{gemination} is predicted. For opaque derivatives with bound roots, \isi{degemination} is predicted.
The suffix \is{-ly}\suffix{ly} is always preceded by a \isi{prosodic word} boundary. It does, however, never form a \isi{prosodic word} on its own. It is thus expected to degeminate. 
As noted above, \cite{Raffelsiefen.1999} additionally states that very frequent derivatives are parsed as a single \isi{prosodic word}. Therefore, an effect of \isi{frequency} on \isi{gemination} is assumed. More frequent derivatives are more likely to degeminate. The predictions are summarized below.




\begin{description}
\item[Prosodic Word \citep{Raffelsiefen.1999}: \is{morphological gemination}Predictions]
\begin{enumerate}[leftmargin=*]
	\item[]
	\item \prefix{un} geminates \\
	The nasal in \prefix{un}prefixed words with a phonological single consonant will be shorter than the nasal in \prefix{un}prefixed words with a phonological double consonant. 
	%The /n/ in \textit{uneven} will be shorter than the /n/ in \textit{unnatural}.	
	
	\item\prefix{in} degeminates in semantically opaque derivatives with a bound root\\
	The nasal in \prefix{in}prefixed words with a phonological single consonant will be as long as the nasal in \prefix{in}prefixed words with a phonological double consonant if  the derivative is semantically opaque and has a bound root as its base.
	%The /n/ in \textit{in} will be of the same length as the /n/ in \textit{inject}. \\
	%The /m/ in \textit{im} will be of the same length as the /n/ in \textit{im}. 
	
	
	
	\item \prefix{in} geminates in \is{semantic transparency}{semantically transparent} derivatives with a word as a base \\
	The nasal in \prefix{in}prefixed words with a phonological single consonant will be shorter than the nasal in \prefix{in}prefixed words with a phonological double consonant if  the derivative is \is{semantic transparency}{semantically transparent} and has a word as its base.
	%The /n/ in \textit{inevident} will be shorter than the /n/ in \textit{innumerous}. \\
	%The /m/ in \textit{implant} will be shorter than the /n/ in \textit{immigrational}. 
	
	
	\item \prefix{dis} degeminates in semantically opaque derivatives with a bound root\\
	The fricative in \prefix{dis}prefixed words with a phonological single consonant will be as long as the fricative in \prefix{dis}prefixed words with a phonological double consonant if  the derivative is semantically opaque and has a bound root as its base.
	%The /s/ in \textit{dis} will be of the same length as the /s/ in \textit{diss}. 
	
	\item \prefix{dis} geminates in \is{semantic transparency}{semantically transparent} derivatives with a word as a base \\
	The fricative in \prefix{dis}prefixed words with a phonological single consonant will be shorter than the fricative in \prefix{dis}prefixed words with a phonological double consonant if  the derivative is\is{semantic transparency}{semantically transparent} and has a word as its base.
	%The /s/ in \textit{disarm} will be shorter than s the /s/ in \textit{dissatisfy}. 
	
	
	\item  \suffix{ly} geminates \\
	The lateral in \suffix{ly}-suffixed words with a phonological single consonant will be shorter than the lateral in \suffix{ly}-suffixed words with a phonological double consonant. 
	%The /l/ in \textit{truly} will shorter than the /l/ in \textit{really}. 
	
	\item Post-lexical factors affect consonant duration.
	
	\item Derivatives with higher token \isi{frequency} will more likely degeminate.
	
\end{enumerate}
\end{description}


\section{Psycholinguistic approaches to morphological processing} \label{Morphological Gemination: Implications for Psycholinguistic Theories of Morphological Processing}

Empirical studies have found that the morphological structure of a derivative influences its \isi{acoustic realization}, and that there thus is an effect of morphology on fine \isi{phonetic detail} (see, for example, \citealt{Sproat.1993b, Cho.2001, Sugahara.2009, Pluymaekers.2010, Smith.2012, LeeKim.2013, Plag.2017}, see also discussion in \sectref{prosodic word}). 
In the previous section, this interaction between morphological and phonetic structure was explained by assuming a mediator between the two levels, i.e. the \isi{phonological word}. However, there are certain restrictions connected with this assumption. Phonological word boundaries are restricted to mirroring categorical differences in morphological structure (for example simplex versus complex words, or weak versus strong boundaries). It follows that the \isi{prosodic word approach} can only explain categorical, i.e. non-gradient, effects of morphological structure on \isi{phonetic detail}. 
In contrast, some psycholinguistic approaches to \isi{morphological processing} expect gradient, probabilistic effects of morphological structure on \isi{phonetic detail}. In general, these approaches assume morphological structure to be gradient and to directly influence the \isi{phonetic realization} of derivatives, i.e. they assume a direct \isi{morpho-phonetic interface}. 

Different from formal linguistic theories, psycholinguistic approaches are\linebreak mainly based on empirical studies. On the one hand, these studies provide empirical support for the assumptions made by the approaches; on the other, differences in the studies' methodologies and outcomes preclude general, uniform theoretical assumptions about the \isi{morpho-phonetic interface}. Assumptions and predictions are less formalized, less stream-lined and less precise than the ones of formal linguistic approaches. The predictions made are rather probabilistic, often revolve around one theoretical concept and often focus on the properties of individual words. 
 \cite{Hay.2001,Hay.2003}, for example, investigated the influence of \isi{decomposability} on the \isi{phonetic realization} of complex words.  She found more \isi{reduction} with less \is{decomposability}decomposable derivatives than with more \is{decomposability}decomposable derivatives. 
\cite{Pluymaekers.2010} investigated the influence of an affix's \isi{informativeness} on its \isi{phonetic realization} and found that the more informative a linguistic unit is, the less phonetic \isi{reduction} is found. 
\cite{Cohen.2014} conducted a study on the influence of an affix's \isi{paradigmatic probability} on its duration. Her results showed that the more probable a suffix is, the shorter the preceding stem is pronounced.

In this study, I will focus on two frequently investigated and discussed concepts in psycholinguistic approaches: \is{decomposability}\newterm{decomposability} and \is{informativeness}\newterm{morphological informativeness}. Below I will lay out how and why they are assumed to influence the \isi{phonetic realization} of affixed words. This discussion includes the discussion of models of word storage and \isi{morphological processing}. Furthermore, I will provide an overview of previous empirical work.
 At the end of each section, I will present how each of the two factors is expected to influence \isi{gemination} with \is{un-}\prefix{un}, \is{in-}\prefix{in}, \is{dis-}\prefix{dis} and \is{-ly}\suffix{ly}.

\subsection{Decomposability} \label{decomposability}

The idea that \isi{decomposability} influences the \isi{phonetic realization} of complex words is closely connected to \isi{dual-route models} of \isi{morphological processing}. These models assume that complex words are simultaneously stored as a whole and in their parts (cf., for example, \citealt{Frauenfelder.1992,Schreuder.2015,deVaan.2011,Caselli.2016}). According to these models, an affixed word like \textit{unnatural} has two separate entries in the \isi{mental lexicon}, the whole-word form (\textit{unnatural}) and the decomposed form, which consists of two separate entries (\is{un-}\prefix{un} and \textit{natural}). Dual-route models stand in opposition to models which assume that a complex word is exclusively stored in its parts (cf., for example, \citealt{Prasada.1993,Marcus.1995,Clahsen.1999,Pinker.2002}). In these models, only irregular and non-transparent lexicalized forms are stored as a whole. Regular complex words, such as \textit{unnatural}, are always stored in their decomposed form and computed online. 

It is generally assumed that the way a word is accessed, i.e. as a whole or via computation, influences its \isi{phonetic realization}. A word that is accessed as a whole will more likely show phonetic \isi{reduction} than a word which is computed online.  This is due to the fact that in \isi{speech processing} morphemes are recognized via their phonological segments, which in turn means that segments that represent morphemes should be more resistant to \isi{reduction}. It follows that models that assume all complex words to be computed online, and all simplex words to be stored, predict systematic differences in the \isi{phonetic realization} of complex and simplex words. Simplex words are expected to show more \isi{reduction}. As discussed above, some empirical studies have indeed found the expected differences (see, for example, \citealt{Cho.2001,Sugahara.2009,Smith.2012}).
These purely computational models, i.e. the models predicting all complex words to be accessed via their parts,  do, however, not predict systematic differences between different complex words. Dual-route models in contrast predict these differences. Complex words accessed via the \isi{whole word route}  are expected to be phonetically more reduced than complex words accessed via the \isi{decomposed route}.



\cite{Hay.2001,Hay.2003} proposes that when accessing a complex word, both routes are activated. She argues that \isi{decomposability} is the main factor governing\linebreak which route is faster in accessing the complex word. The assumption is that the more \is{decomposability}decomposable a complex word is, i.e. the more easily segmentable it is, the more likely it is accessed via its individual parts. The less \is{decomposability}decomposable a complex word is, i.e.  the less easily it is to segment, the more likely it is accessed as a whole. \cite{Hay.2001,Hay.2003} suggests \is{relative frequency}\newterm{relative frequency} as the central measurement for \isi{decomposability}. Relative \isi{frequency} is defined as the \isi{frequency} of a derivative relative to the \isi{frequency} of its base. If a derivative is less frequent than its base, the derivative is rather \is{decomposability}decomposable and thus likely to be accessed via the \isi{decomposed route} (e.g. \textit{impossible, softly}).  If a derivative is more frequent than its base, it is less \is{decomposability}decomposable and thus likely to be accessed as a whole (e.g. \textit{inject, swiftly}). 
The influence of \isi{relative frequency} on the access route is explained by resting activations. A lexical entry of high \isi{frequency} has a higher resting activation than an entry that is less frequently accessed. Higher resting activations lead to faster access, which in turn means that the route which is more frequently used will win the race when accessing a complex word. In other words, if the derivative is more frequent than its base, its resting activation is higher and the \isi{whole word route} is faster. If the base \isi{frequency} is higher than the whole word \isi{frequency}, the \isi{decomposed route} has a higher resting activation and wins the race.


\begin{figure}
\centering
 \begin{tikzpicture}
 	\node at (0,0) (unnatural) [unnatural] {unnatural};
 	\node (derivative) [derivative, below=4cm of unnatural] {\textit{unnatural}};
 	\node (un) [un, below left=2cm of unnatural] {\prefix{un}};
 	\node (natural) [natural, below right=2cm of unnatural] {natural};
 	\draw [arrowN] (derivative) --(un);
 	\draw [arrowN] (derivative) --(natural);
 	\draw [arrowN] (un) --(unnatural);
 	\draw [arrowN] (natural) --(unnatural);
 	\draw [overlay,->, dashed, line width= 0.05 cm] (derivative.east) to[bend right=90,looseness=2.25] (unnatural.east); 	
 	\end{tikzpicture}
 	\caption{Dual-route model as in \cite{Hay.2001}\label{fig:Dual-route model}} 
 \end{figure}
 
\figref{fig:Dual-route model} schematically depicts the race between the two access routes for the word \textit{unnatural}. The dashed arrow indicates the direct route and the solid arrows depict the \isi{decomposed route}. The nodes represent the lexical entries for \textit{unnatural}, and the line width of a node indicates its activation level\footnote{Note that the activation level for the prefix \is{un-}\prefix{un} is not depicted in the picture. The role of the affix in the model will be discussed later in the chapter.}. A very frequent lexeme has a high resting activation (indicated by a thick line width). In our example, the \isi{frequency} of \textit{natural} is much higher than the \isi{frequency} of \textit{unnatural} (72.451 vs. 2.025 in \is{Corpus of Contemporary American English (COCA)} {COCA}). This indicates that the resting activation of \textit{natural} is higher than the one of \textit{unnatural}. Therefore, it is assumed that the base \textit{natural} is accessed faster than the derivative \textit{unnatural}. In other words, the \isi{decomposed route} is accessed faster than the \isi{whole word route}. Note that the difference in \is{frequency}frequencies, i.e. a much higher base \isi{frequency} than derivative \isi{frequency}, indicates that \textit{unnatural} is a highly \is{decomposability}decomposable derivative. We can thus summarize that, according to the decomposability approach, the highly \is{decomposability}decomposable word \textit{unnatural} is predicted to be accessed via its parts and not via the \isi{whole word route}.


As explained above, the route of access is assumed to be mirrored in the \isi{phonetic realization} of a complex word. A word accessed via its parts is expected to display less \isi{reduction} than a word accessed as a whole. Reduction is particularly expected at the morphological boundary. With regard to \isi{morphological gemination}, a cross-boundary phenomenon, one can thus expect words which are less \is{decomposability}decomposable, and which are thus accessed as a whole, to be likely to degeminate, i.e. to show \isi{reduction}. More \is{decomposability}decomposable words are expected to geminate, i.e. show no or less \isi{reduction}. 
Turning back to the example above, one would thus expect \isi{gemination} with \textit{unnatural}. Its high \isi{decomposability}, and consequently the \isi{decomposed route} via which it is accessed, suggests a low degree of \isi{reduction} of the \is{morphological gemination}morphological geminate.


% empirical evidence
Empirical evidence for the effect of \isi{decomposability} on the realization of complex words is contradictory and not conclusive. In the following, I will discuss six studies which investigated \isi{decomposability} and found mixed results. The studies point at several complications with the approach. These complications have to be addressed in order to test the role of \isi{decomposability} in \isi{morphological gemination}, and in \isi{morphological processing} in general. 

 \cite{Hay.2003} investigated  base-final /t/-\isi{reduction} in \is{-ly}\suffix{ly}-suffixed words. By comparing word pairs, she tested whether the stop is more reduced in less \is{decomposability}decomposable words than in more \is{decomposability}decomposable words. In the five word pairs she compared, one derivative was more \is{decomposability}decomposable than the other, i.e. one derivative was less frequent than its base (e.g. \textit{softly}) and one was more frequent than its base (e.g. \textit{swiftly}). The less \is{decomposability}decomposable words were expected to show more deletion and shorter durations.  \citeauthor{Hay.2003} found the expected effect. 
 However, as point-ed out by \cite{Hanique.2012}, \citeauthor{Hay.2003}'s methodology poses some problems. The number of investigated types is very small and hence hardly sufficient to draw general conclusions. Furthermore, instead of testing the direct effect of \isi{decomposability} on duration, Hay compared average durations using a ranking system. This comparison does not allow for a straightforward interpretation. It is thus questionable how robust the effect found really is.

In \citet{Hay.2007},  \is{un-}\prefix{un}prefixed words were investigated. It was tested whether more \is{decomposability}decomposable derivatives have a longer prefix duration than less \is{decomposability}decom\-posable  derivatives.  In contrast to \cite{Hay.2003}, this corpus study tested the influence of \isi{relative frequency} on duration directly. The speech of two groups of speakers was investigated in the study: the speech of \textit{early speakers}, who were born before 1920, and the speech of \textit{late speakers}, who were born after 1920. 
While the expected effect of \isi{relative frequency} on prefix duration was found for the early speakers, there was no effect  for the late speakers. Hay's explanation for the null result with late speakers is that those speakers use \is{un-}\prefix{un} less productively. Therefore, she claims, the prefix is less \is{decomposability}decomposable for late speakers in general. It is thus assumed that the \isi{segmentability} of the prefix, which is not directly accounted for in \isi{relative frequency}, interferes with the effect of \isi{relative frequency}. In other words, the gradient measure of \isi{decomposability} alone, i.e. \isi{relative frequency}, might not be able to capture a derivative's \isi{decomposability}, and it might be necessary to also take an affix's average \isi{segmentability} into account.
 
 \cite{Collie.2008} conducted a study on the effect of \isi{relative frequency} on \isi{stress preservation} with suffixes. While it was expected that derivatives which are accessed via the \isi{whole word route} are less likely to maintain their prosodic structure, complex words accessed via the \isi{decomposed route} were assumed to preserve their prosodic structure by demoting primary \isi{stress} to secondary \isi{stress} (e.g. \textit{acˈcelerate} --  \textit{acˌceleˈration}). Derivatives which are more frequent than their base, i.e. derivatives of low \isi{decomposability}, were thus expected to exhibit non-preserving behavior. While \cite{Collie.2008} found the expected effect for the suffix \suffix{ion}, the effect was not found for the suffix \suffix{ity}. This might mean that only some affixes are affected by gradient \isi{decomposability}, i.e. \isi{relative frequency}.
 	
 For Dutch, \cite{Hanique.2011} investigated the influence of \isi{relative frequency} on the duration of schwa in prefix-final position. Three different prefixes (\prefix{ge}, \prefix{be} and \prefix{ver}) were investigated. Only for one of the three prefixes \isi{relative frequency} showed a significant effect. For \prefix{ge} a higher \isi{relative frequency}, i.e. a higher derivative \isi{frequency} and a lower base \isi{frequency}, led to shorter schwa durations.
 \cite{Hanique.2011} explain the differences in results between the prefixes by referring to differences in \isi{semantic opacity}. While they analyze \prefix{ge} to be \is{semantic transparency}semantically transparent, the other two prefixes are analyzed as semantically opaque. It is suggested that \isi{relative frequency} only affects transparent affixes since opaque derivatives are always  retrieved as a whole from the \isi{mental lexicon}. Similar to \cite{Collie.2008}, this study thus reveals that not all affixes are affected by \isi{relative frequency} and that categorical factors of \isi{decomposability}, such as \isi{semantic transparency}, might interfere with gradient effects of \isi{decomposability}. 


That \isi{relative frequency} does not have the same effect for all affixes is also shown by \cite{Schuppler.2012}. In their data set on Dutch complex words, derivatives with a higher \isi{relative frequency}, i.e. a lower \isi{decomposability}, show less word-final /t/-deletion than derivatives with a lower \isi{relative frequency}, i.e. a higher \isi{decomposability}. In other words, less \isi{reduction} is found with less \is{decomposability}decomposable derivatives. This result goes counter the assumption that less \is{decomposability}decomposable words show more \isi{reduction}. \cite{Schuppler.2012} explain their findings with reference to the \isi{informativeness} of the affix, an idea further discussed in the next section.

The five studies mentioned above primarily concentrated on \isi{relative frequency} as a measurement of a derivative's \isi{decomposability}. The results reveal, however, that other measures of \isi{decomposability}, i.e. \isi{semantic transparency} (cf. \citealt{Hanique.2011}) and the \isi{productivity} of the affix (cf. \citealt{Hay.2007}), might also be predictive for the realization of complex words.  
\cite{Burki.2011} investigated yet another measure of \isi{decomposability}. They used a decomposability rating to predict schwa \isi{reduction} in French complex words. Words rated as being less \is{decomposability}decomposable were expected to show more \isi{reduction}. No effects were found. There are several possible explanations for the null result. 
The first (and most straightforward) explanation is that there is no effect of \isi{decomposability} on schwa \isi{reduction}. This explanation needs to be tested by further research. 
The second explanation is that a rating is not a suitable measure of \isi{decomposability}. However, contra this explanation, \cite{Hay.2001,Hay.2003} showed that there is a significant correlation between decomposability ratings and \isi{relative frequency}. If \isi{relative frequency} is a suitable measure of \isi{decomposability}, so should be decomposability ratings. It might, however, be the case that one of the two measures is a better predictor for durations in a particular type of model.  Differences in distributions and fine-grainedness of the measure might cause different effect sizes, as well as differences in significance between measures.   
The third explanation is also related to methodological aspects. In \cite{Burki.2011}, mean ratings were used and the authors explicitly mention the skewed distribution of the ratings, which, as just explained, might have influenced the results. Furthermore, only the ratings of five participants were included. This might have affected the validity of the rating in general. 
The question whether decomposability ratings are suitable to operationalize \isi{decomposability} thus remains open at this point.


% summary results
The review of the empirical work on \isi{decomposability} has revealed that the empirical facts are unclear and inconclusive. Some studies have found the expected effect of \isi{decomposability} and some have not, one has even found the opposite effect (\citealt{Schuppler.2012}). 
One major reason for the incongruities between studies can be seen in their diverging methodologies. Two important issues, which are also relevant for the present investigation, have to be addressed  here. 
First, the studies investigated different phenomena in different domains (e.g. segment deletion in the base, affix duration, \isi{stress shift}), and it might be that there are differences with regard to the influence of \isi{decomposability} depending on the domain investigated. 
Second, the operationalization of \isi{decomposability} varied across studies. This problem is also discussed by \citet[16]{Hanique.2012}, who state there is no uniform definition of \isi{decomposability} and that ``[f]urther studies have to provide a better definition of \is{decomposability}morphological decomposability before we further investigate the role of morphological structure in speech production''. 
The most frequently used measure of \isi{decomposability} is \isi{relative frequency}, defined as the \isi{frequency} of the derivative relative to its base. But even this seemingly well-defined measure was applied in different ways across studies. While some studies used it as a gradient measure (e.g. \citealt{Hanique.2011,Schuppler.2012}), in others it was used in categorical terms. \cite{Hay.2001} and \cite{Collie.2008}, for example, compared more \is{decomposability}decomposable with less \is{decomposability}decomposable words, i.e. instead of testing the gradient effect of \isi{relative frequency} on \isi{phonetic realization}, they redefined \isi{relative frequency} as a binary measure. In addition to \isi{relative frequency},  the studies referred to \isi{semantic transparency} (cf. \citealt{Schuppler.2012}), \isi{productivity} (cf. \citealt{Hay.2007}) and a decomposability rating (cf. \citealt{Burki.2011}) as measures of \isi{decomposability}. Furthermore, one can think of additional operationalizations of \isi{decomposability}. \cite{Hay.2003}, for example, shows that phonological transparency correlates with \isi{relative frequency}. Additionally, the type of base of a derivative or semantic similarity measures might be used.  

%affix specific
A second reason for differences in results might be related to the affix under investigation, as also suggested by some of the authors themselves (e.g. \citealt{Hanique.2011,Schuppler.2012}). It might be the case that only certain affixes are affected by certain \is{decomposability measure}decomposability measures (cf. \citealt{Collie.2008,Hanique.2011,Schuppler.2012}). It might, for example, be that only very productive, transparent affixes are affected by the a derivative's particular \isi{decomposability} (as suggested by \citealt{Hanique.2011}). This idea is supported by \citeauthor{Hay.2007}'s (2007) data, in which prefix duration is only affected by \isi{relative frequency} when the prefix is productive. The \isi{informativeness} of an affix might also play a role, as suggested by \cite{Schuppler.2012} (see also next section fur further discussion). 
The idea that only transparent affixes are influenced by \isi{decomposability} entails the assumption that only derivatives with transparent affixes can be accessed via the \isi{decomposed route},  i.e. that derivatives with opaque affixes will always be accessed as a whole. This in turn would mean that the predictions made by the decomposability approach are not only word-specific but must also concern the affixes involved.



%Thus prediction made by the approach not that easy to test
At first sight, the predictions made by the decomposability approach seemed quite clear: in complex words of high \isi{decomposability} the \is{morphological gemination}morphological geminate will geminate, and in complex words of low \isi{decomposability} the \is{morphological gemination}morphological geminate will degeminate. The discussion above has, however, shown that after all, the predictions are not that straightforward.
 How can one test the influence of \isi{decomposability}, i.e. operationalize the concept? Which role does the affix play? Furthermore, one might raise the question of whether, according to the decomposability approach, \isi{degemination} is predicted to be categorical or gradient. 




 To address the problem of how to operationalize \isi{decomposability}, I will include five possible measures of \isi{decomposability} in my studies: \isi{relative frequency}, \isi{semantic transparency}, type of base, a decomposability rating and semantic similarity scores. I will first investigate their relation to each other in order to validate their assumed correlation, i.e. ensure that they all tap into the same underlying concept. I will then test the predictions for \isi{gemination} using the different measures. I will thus find out which variable is best suited to predict phonetic \isi{reduction} in terms of duration.



Let us now turn to the role of the affix. In addition to investigating \is{decomposability}word-specific decomposability, I will also test affix-specific \isi{decomposability}. There are two possible explanations of why we might find affix-specific effects. 
First, it might be the case that derivatives of one affix are so similar in their \isi{decomposability} that practically all derivatives behave uniformly with regard to \isi{gemination}. 
The second possibility is that, as suggested by previous research, only some of the affixes under investigation are affected by \isi{decomposability}. To explore these possibilities, it is necessary to investigate the \isi{segmentability} of the five affixes under investigation. 


\begin{table}[h]
\resizebox{\textwidth}{!}{\begin{tabular}{lll}
\lsptoprule
	& {Segmentability}&	{Additional 	}    \\ &	{hierarchy	}	&		{assumption }    \\\midrule
	{Semantic} & \is{un-}\prefix{un} > \{\is{dis-}\prefix{dis}, \is{in-}\prefix{in}\textsubscript{\textsc{Neg}}\}>  \is{in-}\prefix{in}\textsubscript{\textsc{Loc}} > \is{-ly}\suffix{ly}& lexical meaning over pro-   \\	
{Hierarchy}	& & ductivity, transparency and    \\ & & type of base   \\\tablevspace
{Non-Semantic}	&  	\is{un-}\prefix{un} > \is{-ly}\suffix{ly} > \{\is{dis-}\prefix{dis}, \is{in-}\prefix{in}\textsubscript{\textsc{Neg}}\}>  \is{in-}\prefix{in}\textsubscript{\textsc{Loc}}& \isi{productivity}, transparency   \\	
{Hierarchy}& & and  type of base	over   \\	
& & lexical meaning    \\	
	\lspbottomrule
\end{tabular}}
	\caption{Lexical segmentability hierarchies of  affixes}
	\label{fig:Lexical segmentability hierarchies of  affixes} 
\end{table}

 In \sectref{comparison affixes} we have already seen that there seem to be systematic differences in \isi{segmentability} between the affixes. 
The two \isi{segmentability} hierarchies which refer to lexical \isi{decomposability}, i.e. the two hierarchies relevant for the decomposability approach, are depicted in \tabref{fig:Lexical segmentability hierarchies of  affixes}. As discussed earlier, the two hierarchies deviate in the placing of the suffix \is{-ly}\suffix{ly}, which is debatable and which depends on the role of semantics in \isi{decomposability} --  an issue which will be discussed further in the next section.
According to the decomposability approach, affixes which are more segmentable, i.e. affixes which are higher on the \isi{segmentability} hierarchy, are expected to geminate. Affixes which are less segmentable, i.e. affixes which are lower on the hierarchy, are expected to degeminate. Furthermore, one might expect differences in the \isi{degree of gemination} depending on the affix's position in the hierarchies. Gemination with more segmentable affixes is expected to be stronger than \isi{gemination} with less segmentable affixes. The strength of \isi{gemination} is expected to be mainly indicated by the durational differences between phonological doubles and phonological singletons. Stronger \isi{gemination} goes together with larger singleton-double ratios.\footnote{Note that I will use the terms \newterm{strength of gemination} and \newterm{degree of gemination} interchangeably in this book. An affix which shows strong \isi{gemination}, geminates to a high degree. An affix which shows weak \isi{gemination}, geminates to a low degree.}




I will test the affix-specific \isi{decomposability} predictions by first validating the \isi{segmentability} status of the five affixes, i.e. I will look at the distributions of the different \is{decomposability measure}decomposability measures across affixes, and will thereby test whether the theoretically-based hierarchies are borne out by the data. I will then test whether there are significant differences in \isi{gemination} behavior between affixes. If so, I will compare these differences with the \isi{segmentability} hierarchies proposed. In other words, I will check whether  the differences in the \isi{degree of gemination} between affixes mirror the differences in their \isi{segmentability}.\largerpage[-1]

% in which a \isi{phonological rule} deletes one segment of the double consonant
Let us now turn the question of the nature of (de)\isi{gemination}. The decomposability approach does not make clear predictions about the nature of \isi{degemination}. Degemination can thus either be gradient, i.e. the double consonant shows more or less \isi{reduction}, or categorical, i.e. in case of \isi{degemination} there is no durational difference between singletons and doubles. %In contrast, the formal approaches discussed regard \isi{degemination} as a categorical process in which one of the two double consonants is deleted by some kinf of \isi{phonological rule}. 
If one assumes \isi{degemination} to be categorical, one simultaneously assumes \isi{gemination} to be governed by categorical factors. For certain categories, phonological doubles are categorically longer than phonological singletons. For others, there is no durational difference between doubles and singletons. 
If one assumes \isi{degemination} to be gradient, one assumes \isi{gemination} to be governed by gradient, word-specific factors.
%In this case, one would expect gradient durational differences between singletons and doubles which depend on word-specific factors.

In this study, I will explore both possibilities by investigating the distribution of durations across singletons and doubles, and by investigating which kind of factors govern \is{morphological gemination}{gemination}. %, i.e. categorical or word-specific factors. 
If \isi{gemination} is categorical, the durations of singletons and doubles should show a bimodal distribution. Doubles should be longer than the singletons. If \isi{gemination} is gradient, one would expect a gradient increase in duration from singletons to doubles, i.e. no binary distribution (cf.  \cite{Hanique.06.03.2013} for a similar analysis of distributions to investigate the nature of schwa \isi{reduction} in Dutch). Furthermore, in case of gradient \isi{gemination}, the durational difference between doubles and singletons should be affected by word-specific factors, such as, for example, \isi{relative frequency}. In case of categorical \isi{gemination}, the durational difference between doubles and singletons should be affected by categorical factors, such as, for example, the affix.
 
To summarize, one can state that up to this point the decomposability approach does not make clear, spelled-out predictions for \isi{gemination}. Before testing the effect of \isi{decomposability} on \is{morphological gemination}{gemination}, it is necessary to explore the concept of \isi{decomposability} with its possible operationalizations and domains. Also, the approach does not make assumptions about the nature of \isi{gemination}, i.e. it is yet to explore whether \isi{gemination} is gradient or categorical. 
For these reasons the predictions uttered at this point remain relatively vague. 

Two different predictions with regard to the effect of \isi{decomposability} are formulated. One concerns the \isi{decomposability} of the individual derivative, and one the \isi{segmentability} of the affix. They will be tested using different \is{decomposability measure}decomposability measures. i.e.  \isi{relative frequency}, \isi{semantic transparency}, type of base, a decomposability rating and semantic similarity scores. The two predictions are spelled out below.\pagebreak

\largerpage
\begin{description}
\item[\isi{Morphological Segmentability Hypothesis}: \is{morphological gemination}Predictions]
\begin{enumerate}[leftmargin=*,label=\Alph*:]
    \item[]
 	\item The \isi{decomposability} of an individual word influences gemination
        \begin{itemize}[leftmargin=*]
        \item The more \is{decomposability}decomposable a derivative is, the higher is its \isi{degree of gemination}.
        \item The less \is{decomposability}decomposable a derivative is, the lower is its \isi{degree of gemination}.
        \end{itemize}
        
    \item The \isi{segmentability} of an affix influences gemination
        \begin{itemize}[leftmargin=*]
        \item The more segmentable an affix is, the higher is the \isi{degree of gemination} with words containing that affix. 
        \item The less segmentable an affix is, the lower is the \isi{degree of gemination} with words containing that affix.
        \end{itemize}
\end{enumerate}
\end{description}

In addition to the two \isi{decomposability} predictions, two predictions with regard to the nature of \isi{gemination} are formulated. One predicts \isi{gemination} to be categorical, the other predicts \isi{gemination} to be gradient. They will be tested by investigating the distribution of durations in the data sets and by investigating the type of effects governing \isi{gemination}. Note that the two predictions concerning the nature of \isi{gemination} are not exclusive to the decomposability approach but concern all approaches discussed. While formal approaches explicitly predict \isi{gemination} to be categorical, the psycholinguistic approaches leave the question open. The two predictions are displayed below.


\begin{description}
\item[Nature of \isi{gemination}: \is{morphological gemination}Predictions] \label{predictions nature of gemination}

\begin{enumerate}[leftmargin=*,label=\Alph*:]
    \item[]
 	\item Gemination is categorical
        \begin{itemize}[leftmargin=*]
            \item The duration of the affixational consonant(s) in the data set shows a bimodal distribution with one mode representing doubles and one mode being singletons.
            \item Doubles are longer than singletons. 
            \item Gemination is governed by categorical factors.
        \end{itemize}
    \item Gemination is gradient
            \begin{itemize}[leftmargin=*]
            \item The duration of the affixational consonant(s) in the data set does not show a bimodal distribution with one mode representing doubles and one mode being singletons.
            \item The duration of the affixational consonant(s) increases gradually\linebreak from singletons to doubles.
            \item Gemination is governed by word-specific factors.
            \end{itemize}
\end{enumerate}
\end{description}

\subsection{Morphological informativeness} \label{morphological informativeness}

The idea that linguistic units with high information load are less prone to phonetic \isi{reduction} than linguistic units with low information load is well established in psycholinguistic approaches  (cf., for example, \citealt{Aylett.2004,Kuperman.2007,Pluymaekers.2010,Hanique.2012}). Following \cite{Lindblom.1990}, in speech production two forces work against each other: economy of articulatory effort and discriminability of the speech signal. On the one hand, speakers want to put as little effort as possible in producing speech, on the other, they want to ensure the intelligibility of the speech signal. As a result, they only put as much effort in pronunciation as they estimate to be necessary for the listener to discriminate the speech signal. A speaker's amount of effort is mirrored in the degree of \isi{reduction} found in the speech signal, and depends on the information load of the linguistic unit. The degree of information load in turn is influenced by various factors, such as probability of occurrence, semantic information load and redundancy (see also \citealt{Kuperman.2007} for discussion).  
Elements with low information load, i.e. redundant elements with higher probabilities of occurrence and less semantic information load, are realized with less effort and are hence more reduced than elements with higher information load. They are shorter and less salient. 

The elements under investigation in this study are affixes. Following the approach just described, one can assume affixes with higher information load to show less \isi{reduction} than affixes with lower information load. Degemination can be defined as some sort of \isi{reduction}. Even though the nature of \isi{gemination} is yet unclear (see \sectref{decomposability} for discussion on gradient vs. categorical \isi{gemination}), it is certain that \isi{degemination} results in some kind of phonetic \isi{reduction}. It is thus predicted that affixes with higher information load are less prone to \isi{degemination} than affixes with lower information load, and that the degree of (de)\isi{gemination} depends on the degree \isi{informativeness} of the affix.

To test this prediction, it is necessary to measure the information load of the affixes.
 In this study, the \isi{predictability} of the affix and the semantic information load of the affix are used as indicators of \isi{informativeness}.
The \isi{predictability} of an affix is closely related to its probability of occurrence. 
 If an affix is probable to occur, it is very predictable and thus not very informative. %In turn, \isi{reduction} is likely. 

 
 Probability can be operationalized in various ways. 
   %Paradigmatic Probability
   \cite{Pluymaekers.2010}, for example, tested the effect \isi{paradigmatic probability} on duration to investigate the effect of \isi{informativeness} on phonetic \isi{reduction}.
 They investigated Dutch \textit{\mbox{-igheid}} (/əxhɛit/), which represents three different morphological structures. In some words \suffix{igheid} represents one single suffix (\textit{\textit{-igheid}-words}), in some \suffix{ig} belongs to the base word and only \suffix{heid} is the suffix (\textit{-heid-words}), and for some words both parsings are possible (\textit{ambiguous words}). \cite{Pluymaekers.2010} measured the \isi{informativeness} of the investigated structures by counting the cohort of competitors in the morphological \is{morphological paradigm}paradigm for each structure. 
 The more competitor words in a \is{morphological paradigm}paradigm, the less probable, and thus the more informative, the structure is.  Since the \is{morphological paradigm}paradigm for \suffix{heid}-words is the least dense, i.e. this suffix has the least competitors, it is the least informative and most \isi{reduction} is expected with words of this kind. Reduction was measured in terms of the duration of the /xh/ cluster. \cite{Pluymaekers.2010} found the expected effect. The structure with least competitors in the morphological \is{morphological paradigm}paradigm, i.e. the least informative structure, showed most \isi{reduction}.
  
  
 Similarly \cite{Schuppler.2012} found a relation between \isi{informativeness} in terms of number of competitors in the morphological \is{morphological paradigm}paradigm and \isi{reduction}. 
 As laid out in the previous section, in their data complex words with a low \isi{relative frequency}, i.e. highly \is{decomposability}decomposable words, showed more base-final /t/ \isi{reduction} than less \is{decomposability}decomposable words. This goes against the assumptions made by the decomposability approach, and is the opposite of what was found for English adverbial \is{-ly}\suffix{ly} in \cite{Hay.2003}. 
 
 \cite{Schuppler.2012} explain the difference between their and Hay's results by referring to differences in \isi{informativeness} between the two investigated structures. 
 They hypothesize that \isi{relative frequency} might play a different role for suffixes with higher information load, i.e. less probable suffixes, than for suffixes with lower information load. According to \cite{Schuppler.2012}, English \is{-ly}\suffix{ly} is more probable and less informative than the Dutch suffix \suffix{t}. 
 English adverbs always end in the suffix \is{-ly}\suffix{ly}, i.e. it has a high \isi{paradigmatic probability}. The suffix is thus expected and does not feature a high information load. It differs in this respect from the inflectional Dutch suffix \suffix{t}, which is only one of three possible forms in the inflectional \is{morphological paradigm}paradigm. According to \cite{Schuppler.2012}, the suffix \suffix{t} is therefore generally less probable than \is{-ly}\suffix{ly}. It has a higher information load and is not reduced. In other words, because Dutch inflectional \suffix{t} is very informative, lower \isi{decomposability} does not lead to more \isi{reduction} with this affix. Lower \isi{decomposability} leads to more \isi{reduction} with the less informative English adverbial suffix  \is{-ly}\suffix{ly}.
 
 
 %Why I don't use it 
The two studies above suggest that paradigmatic probabilities might be a good measure of \isi{predictability}. However, this measure is not applicable in this study. The reason is that it is yet unclear how to measure the \isi{paradigmatic probability} of derivational prefixes. 
 To nevertheless investigate the \isi{predictability} of the affixes in this study, 
another type of probability was looked at: \isi{syntagmatic probability}. 

As shown in \cite{Hanique.06.03.2013}, \isi{syntagmatic probability} can also successfully be used as an indicator of \isi{predictability} and \isi{informativeness}. In \cite{Hanique.06.03.2013}, the deletion of word-final /t/ in Dutch past participles and in Dutch simplex words was investigated.  The segment was more often deleted in complex than in simplex forms. 
  \cite{Hanique.2012}  argue that the results can best be accounted for by reference to the \isi{informativeness} of the suffix \suffix{t} in comparison to the \isi{informativeness} of the segment /t/ in corresponding simplex structures. Since most Dutch participles end in /t/, the segment is highly predictable in complex words. Therefore, it is less informative than /t/ in corresponding simplex words. The higher \isi{reduction} rate of /t/ in complex words can thus be explained by the low degree of \isi{informativeness} of the suffix \suffix{t}.
  
  
Let us now turn to the affixes under investigation in this study. Out of the five investigated affixes, the suffix \is{-ly}\suffix{ly} features the highest \isi{syntagmatic probability}. Due to its function to create adverbs, its \isi{syntagmatic probability} is very high, and, in turn, its \isi{predictability} is very high. As the prefixes are not associated with a specific function, they are much less predictable.  
Furthermore, the suffix \is{-ly}\suffix{ly} is more predictable than the prefixes because of its position at the end of the word. 
 Prefixes precede their base, which means that the base of prefixes does not serve as a cue for the occurrence of the prefix. Prefixes can basically occur after any word after which its base can occur. 
 For example, the prefix \is{un-}\prefix{un} as in \textit{uncool} can occur after any word after which the word \textit{cool} can occur. 
 In contrast, suffixes follow their base. Their occurrence is restricted by the amount of base words they can take,  and their base serves as a cue for their occurrence.
 It follows that the \isi{syntagmatic probability} of prefixes, and in turn their \isi{predictability}, is generally lower than the one of suffixes. 
 
Overall, the discussion of \isi{predictability} has shown that the suffix \is{-ly}\suffix{ly} is the most predictable affix in this study. It is much more predictable than the prefixes in this study. The differences in \isi{predictability} between the prefixes is less clear. As explained above, paradigmatic probabilities are not useful in order to measure a prefix's \isi{predictability}. Furthermore, it seems quite challenging to assess differences in the \isi{syntagmatic probability} of derivational prefixes. As prefixes occur before their base, and as they are not associated with a specific function, it is unclear on which base one can compare their probability of occurrence.\largerpage




%Semantic Information Load

In addition to \isi{predictability}, the semantic information load of an affix can also serve as a measure of its \isi{informativeness}. An affix that contributes more to the meaning of a derivative is more informative than an affix which does not feature clear semantic content. Semantic information load is particularly interesting as a measure of \isi{informativeness} for derivational prefixes (as the ones in this study), for which the application of other \isi{informativeness} measures is quite problematic. 

The semantics of the affixes under investigation were discussed thoroughly in \sectref{comparison affixes}. While some affixes, such as \is{un-}\prefix{un}, feature a stable, transparent meaning, others, like locative \is{in-}\prefix{in} and \is{-ly}\suffix{ly}, do not. The Semantic Segmentability Hierarchy in \tabref{fig:Lexical segmentability hierarchies of  affixes} depicts the decline in semantic information load of the five investigated affixes.
 The lower the affix is positioned on the hierarchy, the less semantic information it conveys, and the less informative it is. The hierarchy shows that the suffix \is{-ly}\suffix{ly} is the least informative affix with regard to its semantic information load.  As discussed earlier, the suffix does not contribute any lexical meaning to the derivative. 
 For the prefixes, \is{un-}\prefix{un} is the prefix with the most transparent and stable meaning, i.e. the affix with the highest information load. Negative \is{in-}\prefix{in} and \is{dis-}\prefix{dis} denote a stable, negative meaning in most derivatives but there are also some derivatives in which the affix does not contribute a clear lexical meaning. Locative \is{in-}\prefix{in} features the least semantic information of the prefixes (see \sectref{affixes} for detailed discussion of the semantics of all five affixes). 

One can summarize that, due to its high \isi{predictability} and low degree of semantic contribution to a derivative's meaning, the suffix \is{-ly}\suffix{ly} is the least informative affix in this study. It is thus expected to show the weakest \isi{degree of gemination}. 
The analysis of the prefixes' \isi{informativeness} is mainly based on semantic factors. 
The analysis revealed that \is{un-}\prefix{un} is the most informative prefix. It is thus expected to show the highest \isi{degree of gemination}. 
The \isi{informativeness} of locative \is{in-}\prefix{in}, negative \is{in-}\prefix{in} and \is{dis-}\prefix{dis} is less clear and might vary among types. In derivatives with transparent meaning, the affixes are more informative than in derivatives with opaque meaning. This means there are two possible predictions for the \isi{gemination} with these three prefixes: an affix-specific one and a word-specific one. 

According to the affix-specific prediction, one would predict the overall \isi{informativeness} of the affix to govern \isi{gemination}. This means one would predict \is{morphological gemination}{gemination} to pattern according to the Semantic Segmentability Hierarchy. While \isi{gemination} with \is{un-}\prefix{un} is expected to be the strongest, and \isi{gemination} with \is{-ly}\suffix{ly} is expected to be the weakest, the other three prefixes are expected to pattern in between. Importantly, one would not predict word-specific effects according to this prediction. 

According to the word-specific prediction, one would predict \isi{informativeness} to be word-specific, i.e. in \is{semantic transparency}semantically transparent words a prefix is more informative than in semantically opaque words. Since \is{un-}\prefix{un} is always \is{semantic transparency}semantically transparent, it is predicted to always display a high \isi{degree of gemination}. For the other three prefixes the \isi{degree of gemination} is expected to depend on \isi{semantic transparency}. Opaque derivatives should display weaker \isi{gemination} than transparent derivatives. 

In this study, I will test both predictions made by the \is{informativeness}morphological informativeness approach, i.e. the affix-specific and the word-specific prediction. The predictions are summarized below.

\begin{description}
\item[Morphological Informativeness: \is{morphological gemination}Predictions]
    \begin{enumerate}[label=\Alph*:,leftmargin=*]
        \item[]
        \item Affix-specific \isi{informativeness} influences gemination
            \begin{itemize}[leftmargin=*]
                \item The more informative an affix is, the higher is the \isi{degree of gemination} with words containing that affix.
                \item The less informative an affix is, the lower is the \isi{degree of gemination} with words containing that affix. 
                \item Gemination patterns according to the Semantic Segmentability Hierarchy.
            \end{itemize}
        \item Word-specific \isi{informativeness} influences gemination
			\begin{itemize}[leftmargin=*]
                \item The more informative an affix is in a given derivative, the higher is the \isi{degree of gemination} in the derivative.
                \item The less informative an affix is in a given derivative, the lower is the \isi{degree of gemination} in the derivative.% words
				\item Derivatives with the prefix \is{un-}\prefix{un} geminate to a high degree.		
				\item Derivatives with the suffix \is{-ly}\suffix{ly} geminate to the lowest degree.
				\item  Derivatives with transparent semantics and the prefixes \is{dis-}\prefix{dis}, negative \is{in-}\prefix{in} and locative \is{in-}\prefix{in} geminate to a high degree.	
				\item  Derivatives with opaque semantics and the prefixes \is{dis-}\prefix{dis}, negative \is{in-}\prefix{in} and locative \is{in-}\prefix{in} geminate to a low degree.
			\end{itemize}
    \end{enumerate}
\end{description}
 
\section{Speech production models}\label{speech production models}
%\enlargethispage{\baselineskip}

Speech production models are closely connected to the approaches discussed in the previous sections. As formal linguistic and psycholinguistic approaches, they also make assumptions about the \isi{morpho-phonological interface} and are thus relevant for this work. However, \is{speech production model}speech production models are broader than the approaches previously discussed in that they are not solely concerned with the \isi{morpho-phonological interface} but with speech production as a whole. In other words, the \isi{morpho-phonological interface}, and the \is{morphological processing}processing of complex words, form just a small part of these models.

 Two types of  \is{speech production model}speech production models can be distinguished: \isi{modular feed-forward models} (cf. \citealt{Levelt.1999,Levelt.1999b,Levelt.2000}) and \isi{usage-based models} (cf. \citealt{Johnson.1997b,Bybee.2002,Pierrehumbert.2001,Pierrehumbert.2002}). With regard to \isi{morphological processing}, the two types differ in their assumptions of how words are stored, as well as which factors influence the \isi{phonetic realization} of morphemes. Crucially, neither is very explicit with regard to the \isi{morpho-phonological interface} and no specific claims about the interplay of phonetics and morphology are made. Therefore, no explicit predictions about \isi{gemination} can be drawn from the models. The data in this study might nevertheless provide evidence for the theoretical modeling of  speech production by displaying general effects on duration. By investigating which morphological factors influence segmental duration at morpheme boundaries, general assumptions about the \isi{morpho-phonological interface} made by different types of \is{speech production model}speech production models can be tested.

In traditional \is{speech production model}speech production models, such as  \cite{Levelt.1999b}, two main stages of processing can be distinguished: the \isi{lexical stage} and the \isi{post-lexical stage}. At the \isi{lexical stage} lemmas are retrieved and grammatically encoded. At the \isi{post-lexical stage} the morpho-phonological and the phonetic encoding take place, i.e. the relation of morphological and phonetic structure is defined at this stage. After all morphemes of a lemma are activated and assembled, their morpho-phonological code is spelled out. This code is segmental in nature. The spelled-out segments are then syllabified to form \newterm{phonological words}.\footnote{Note that the term \is{phonological word}\textit{phonological word} in \cite{Levelt.1999b} is not synonymous with the term in the \isi{prosodic word approach} (cf. \sectref{prosodic word}). } In a last step utterance prosody is generated to compute the \newterm{phonological score}, which serves as the basis for articulation. In other words, the phonological score is translated into phonetics, which is in turn translated into concrete instructions for the articulators (\newterm{articulatory score}).

%\newpage
The crucial point with regard to morpho-phonological processing is that articulation is based on \is{phonemic representation}{phonemic representations}, i.e. the segmental morpho-phono-logical code. The question is whether morphological structure is present at this stage, and if so, how it is mirrored in the \isi{acoustic realization} of complex words.
 As discussed by \citet[1037]{CohenGoldberg.2013}, most traditional models, such as the one just described or the one suggested by \cite{Dell.1986}, are largely silent about the \is{post-lexical level}{post-lexical processing} of multi-morphemic words. It is simply not stated whether the assembly of morphemes leaves traces in the phonological make-up of a word. However, since none of the models mentions any process which suggests that morphological structure is preserved in morpho-phonological encoding, it can be assumed that \is{phonemic representation}{phonemic representations} do not feature any traces of morphological structure. This also means that, according to traditional \is{speech production model}{models of speech production}, there is no difference in the \isi{phonemic representation} of morphologically complex and morphologically simplex words, i.e. there is no difference in their \isi{acoustic realization}.

Recent studies have challenged this assumption by showing that phonemically identical strings of different morphological status vary systematically in their \isi{phonetic realization}. For example, \cite{Kemps.2005} and \citet{Blazej.2015} found that phonologically identical free and bound variants of a base (e.g. \textit{clue} without a suffix compared to \textit{clue} in \textit{clueless}) differ acoustically.
As already discussed in \sectref{prosodic word}, other studies demonstrate that the realization of segments can vary systematically depending on the type of boundary (affix, compound, phrase) they are adjacent to  (e.g. \citealt{Sproat.1993b, Smith.2012,LeeKim.2013}). Furthermore, empirical work found systematic durational differences between \is{homophone}{homophonous} affixes. \cite{Plag.2017} and \cite{Godfrey.2016}, for example, found \is{homophone}{homophonous} English suffixes to display systematic differences in duration.
 
 The studies above thus challenge standard \is{speech production model}{models of speech production} and demand for modifications. These modifications must explain the effect of morphological structure on the \isi{acoustic realization} of complex words.  
 \cite{CohenGoldberg.2013}, for example, suggests an extension of standard theories by proposing the \is{heterogeneity of processing hypothesis}\newterm{heterogeneity of processing hypothesis}. As a consequence of the morpheme assembly, the hypothesis predicts structural weaknesses at morphological boundaries of \is{phonemic representation}phonemic representations. Each morpheme acts as an independent domain for \is{post-lexical stage}{post-lexical processes}, which will therefore apply more strongly to tautomorphemic phonemes than heteromorphemic phonemes. Heteromorphemic phonemes are predicted to be less integrated with each other than tautomorphemic phonemes. Furthermore, it is proposed that the phonemes in multimorphemic words will inherit the lexical properties of the morpheme they belong to, i.e. there will be differences in activation levels between different morphemes of one word (e.g. because of different \is{frequency}frequencies of the constituents of the complex word). It is therefore predicted that those aspects of \is{post-lexical stage}{post-lexical processing} that are influenced by lexical properties (e.g. duration, vowel space) will vary by morphemes. 
 However, as noted by the author himself, the hypothesis is not fully specified yet. It calls for more empirical work, and must be elaborated to not only account for differences between simplex and complex words but also for differences between morphemes of varying \isi{boundary strength} (cf. \citealt[1057f.]{CohenGoldberg.2013}).
 
 
 %usage based models
 Instead of modifying traditional models, one could also assume another type of speech production model, e.g. \isi{usage-based models} that assume a direct \isi{morpho-phonetic interface}. These models might be better suited to explain the phonetic implementation of morphologically complex words than traditional models. Exemplarbased models (cf. \citealt{Johnson.1997b,Pierrehumbert.2001,Pierrehumbert.2002}; \citealt{Bybee.2002}), for example, assume that the \isi{phonetic realization} of a word, simplex and complex, is determined by exemplars experienced by the speaker. It is assumed that all phonetic variants of a word are stored as exemplars in a speaker's memory. These exemplars are organized in a network structure. When producing a word similar exemplars are activated, and every activated exemplar influences the target production. More frequent exemplars have stronger representations and are therefore expected to influence pronunciation to a higher degree than less frequent exemplars. 
 Since, in contrast to traditional models, word-specific, fine phonetic  information is stored in these models, differences in the realization of words with different morphological structure are expected. Systematic differences might be explained by the proposed network structure.
 However, \isi{usage-based models} are not very specific with regard to the network structure, i.e. they do not explicitly define at which levels similarities play a role and how these different levels interact. Furthermore, according to exemplar-based models, speech production is a speaker-specific process, i.e. differences in phonetic realizations can always be explained by differences in exemplar-structure between speakers. To conclude, up to now \isi{usage-based models} do not make explicit predictions for the realization of complex words. Further specifications of the models are necessary to test their validity.
 

One can summarize that currently none of the proposed \is{speech production model}speech production models is able to model the \isi{phonetic realization} of complex words, i.e. the relation of morphological structure and fine \isi{phonetic detail}.  Traditional models (e.g. \citealt{Dell.1986,Levelt.1999b}) are silent about the post-lexical \is{morphological processing}processing of complex words. Hypotheses which try to specify \is{post-lexical stage}{post-lexical processing}, such as the \isi{heterogeneity of processing hypothesis}, need further development to be able to account for different acoustic realizations of complex words. The same is true for \isi{usage-based models}, which also need further specification. To accurately model the \is{morphological processing}processing of complex words, it is necessary to conduct further studies. 
These studies might provide some new insight about which factors influence the 
\isi{phonetic realization} of complex words, i.e. which factors must be incorporated in \is{speech production model}{models of speech production}.
 
 

 
 
 %the present study
 The present study can contribute new empirical facts about the role of morphological structure in \isi{phonetic realization} by investigating the \isi{acoustic realization} of the two \is{homophone}{homophonous} prefixes locative and negative \is{in-}\prefix{in}. According to standard \is{speech production model}{models of speech production}, they should behave similarly with regard to their phonetic implementation. There should be no systematic difference in duration between the two affixes, and they should display the same \isi{gemination} behavior. Systematic differences between the two prefixes would provide another piece of evidence for the presence of morphological structure in \isi{phonetic detail}, and would thus support the claim for a revision of standard \is{speech production model}speech production models. These revisions could, for example, pick up and further develop \citeauthor{CohenGoldberg.2013}'s  \isi{heterogeneity of processing hypothesis}. The results could also be used to further specify usage-based \is{speech production model}{models of speech production}.



 
 




\section{Summary: Theoretical implications}\label{summary predictions}

 In this chapter we have seen that the pattern of \isi{morphological gemination} in En\-glish affixation has important implications for various theories of the morpho-phonological and \isi{morpho-phonetic interface}. Even though the different ap-\linebreak proaches deviate from each other in important respects, one can state that all of them are based on the assumption that \is{boundary strength}morphological boundary strength influences the phonetic implementation of complex words. Derivatives with stronger boundaries, i.e. more \is{decomposability}decomposable derivatives, are less likely to be reduced, i.e. are likely to geminate. Derivatives with weaker boundaries, i.e. less \is{decomposability}decomposable derivatives, are more likely to be reduced, i.e. are more likely to degeminate. 
 

 The conceptualization of \isi{boundary strength} deviates, however, vastly between the approaches. While some assume a categorical difference between affixes (e.g. \isi{Lexical Phonology}, \is{Stratal Optimality Theory}Stratal OT), others assume \isi{boundary strength} to be a gradient, probabilistic word-specific concept (e.g. the Decomposability Approach, the Morphological Informativeness Approach). 
 While some approaches define \isi{boundary strength} by means of mainly lexical factors, such as the type of base an affix takes or an affix's \isi{productivity} (e.g. \isi{Lexical Phonology}, \is{Stratal Optimality Theory}Stratal OT), others mainly focus on prosodic aspects (e.g. \isi{Prosodic Phonology}) and others concentrate on semantics (e.g. Morphological Informativeness).  The differences in the conceptualization of \isi{boundary strength} mirror general differences in theoretical assumptions about the \isi{morpho-phonological interface}. As described in detail in the previous sections, these differences lead to different predictions about \is{morphological gemination}{gemination} with the affixes under investigation in this study.
 Testing which approach makes the most accurate predictions can therefore provide an important theoretical contribution. 
 By conducting empirical studies, and thus finding out which factors govern \is{morphological gemination}{gemination} in English affixation, the predictions made by some of the approaches will be falsified, while others will be supported. Note that some outcomes might simultaneously support two approaches. For example, the \isi{degemination} of the suffix \is{-ly}\suffix{ly} is predicted by the \isi{prosodic word approach}, as well as by the \is{informativeness}morphological informativeness approach. And the \isi{gemination} of \is{un-}\prefix{un} is predicted by almost all of the approaches.



\begin{table}
	\caption{Summary of concepts and factors predicting gemination according to different theoretical approaches\label{tbl:Factors predicting gemination}}
		\resizebox{\textwidth}{!}{\begin{tabular}{lll}
				\lsptoprule
			{Approach }& {Concept}& {Factor(s)} \\ \midrule			
			\isi{Lexical Phonology} 									& stratum of affix &affix\\\tablevspace
			Stratal OT 													& stratum of affix& affix\\ 
			&type of base for & type of base\\ 
			&\is{dual-level affix}dual-level affixes& \\\tablevspace
			Prosodic Word										 & \isi{prosodic word} status & affix \\ 
			& 						of affix								&  \isi{semantic transparency}\\ 
			& 														& type of base \\\tablevspace
			Morphological Segment- &\isi{decomposability} of &  \isi{relative frequency} \\ 
			ability (word-specific)															& 	derivative 													& \isi{semantic transparency} \\
			&														& type of base \\
			&														& decomposability rating\\
			&														& semantic similarity\\\tablevspace
			Morphological Segment-&\isi{segmentability} &   affix\\	
			ability (affix-specific) &of affix	& \\\tablevspace
			Morphological Informative-& word-specific  & \isi{semantic transparency}\\
			ness (word-specific) &\isi{informativeness}& \\\tablevspace
			Morphological Informative- & affix-specific &  affix \\
			ness (affix-specific) &\isi{informativeness}& \\
			\lspbottomrule
		\end{tabular}}
\end{table}


\tabref{tbl:Factors predicting gemination} summarizes the approaches discussed. The first column names the approach, the second the theoretical concept assumed to govern \is{morphological gemination}{gemination} and the third the main factor(s) assumed to influence \is{morphological gemination}{gemination}. To test the predictions of each approach, it is crucial to empirically investigate the factors in the third column, i.e. to test their effect on \isi{morphological gemination}.
For the two psycholinguistic approaches discussed, i.e. Morphological Segmentability and Morphological Informativeness, two different predictions were formed: a word-specific one and an affix-specific one. Both are summarized in the table. Note that the affix-specific predictions will be tested by consulting the lexical \isi{segmentability} hierarchies formed in \sectref{comparison affixes}. These hierarchies are based on qualitative analyses of the affixes' features and display the affixes' \isi{segmentability}, as well as their degree of \isi{informativeness} in terms of their semantics. In the course of this book these theoretically formed hierarchies will be verified by empirical data. For convenience, the pertinent hierarchies are repeated in \tabref{fig:Lexical segmentability hierarchies of  affixes repetition}.  

\begin{table}
	\resizebox{\textwidth}{!}{\begin{tabular}{lll}
		\lsptoprule
		& {Segmentability}&	{Additional 	}  		  \\
		&	{hierarchy	}	&		{assumption }  	  \\		
		\midrule
		{Semantic} & \is{un-}\prefix{un} > \{\is{dis-}\prefix{dis}, \is{in-}\prefix{in}\textsubscript{\textsc{Neg}}\}>  \is{in-}\prefix{in}\textsubscript{\textsc{Loc}} > \is{-ly}\suffix{ly}& lexical meaning over pro-	 		  \\	
		{Hierarchy}	& & ductivity, transparency and 	 		  \\	
		& & type of base			 		  \\\tablevspace
		{Non-Semantic}	&  	\is{un-}\prefix{un} > \is{-ly}\suffix{ly} > \{\is{dis-}\prefix{dis}, \is{in-}\prefix{in}\textsubscript{\textsc{Neg}}\}>  \is{in-}\prefix{in}\textsubscript{\textsc{Loc}}&		 \isi{productivity}, transparency\\	
		{Hierarchy}& & and  type of base	over   \\	
		& & lexical meaning\\
		\lspbottomrule
	\end{tabular}}
 	\caption{Lexical segmentability hierarchies of  affixes}
 	\label{fig:Lexical segmentability hierarchies of  affixes repetition} 
 \end{table}


In addition to the factors influencing \isi{gemination} and \isi{degemination}, the nature of \isi{gemination} will be investigated in this study. While the formal linguistic approaches assume \isi{gemination} to be categorical, psycholinguistic approaches do not make assumptions about the nature of the phenomenon, i.e. according to psycholinguistic approaches \isi{degemination} might be gradient. As laid out in detail in \sectref{decomposability}, I will test both possibilities, i.e. I will test the prediction that \isi{gemination} is categorical and the prediction that \isi{degemination} is gradient. This will be done by investigating the distribution of duration across doubles and singletons, and by investigating which type of factors govern \isi{gemination}.

In the last part of the chapter, I discussed implications for \is{speech production model}speech production models. It was shown that the \isi{acoustic realization} of the \is{homophone}{homophonous} prefixes negative and locative \is{in-}\prefix{in} has important implications for the modeling the \isi{acoustic realization} of complex words. Currently \is{speech production model}{models of speech production} (cf. \citealt{Dell.1986,Johnson.1997b,Levelt.1999b,Bybee.2002,Pierrehumbert.2001,Pierrehumbert.2002}) are unspecified with regard to the \is{morphological processing}processing of complex words. Finding differences in the \isi{acoustic realization} of the \is{homophone}{homophonous} \is{in-}\prefix{in}prefixes would contribute further evidence for the presence of morphological structure in \isi{phonetic detail}. Models of speech production would need to be revised, or specified, with regard to this aspect.
